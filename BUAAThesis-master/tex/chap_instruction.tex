
%-------------相关基本理论章节----------------
\chapter{相关基本理论}
\section{信用评价的相关理论}
\subsection{信用评价的目的}
信用评价(Credit Evaluation)也称为信用评估、信用评级、资信评估、资信评级,是以一套相关指标体系为考量基础,标示出个人或企业偿付其债务能力和意愿的过程。信用评价是一个企业履约状况和偿债能力综合反映。
这种评估基于一系列因素,包括财务健康状况、历史偿还记录、业务稳定性、管理质量、市场地位等。信用评价的目的是帮助贷款方、投资者和其他利益相关者了解借款方的信用风险,从而做出更明智的决策。

信用评价通常分为不同的等级,不同的信用评级机构采用不同的评估标准、方法和模型来进行信用评价,也会采用不同的等级划分标准和命名方式。信用评价可以由专门的信用评级机构(如穆迪、标准普尔、惠誉等)进行,也可以由银行、保险公司等金融机构内部进行。

信用等级并不是绝对的,它会随着市场环境、企业经营状况等因素的变化而发生变化。投资者或者企业在参考信用等级时,需要了解评级机构的评估标准和方法,并结合实际情况进行综合分析。
 
\subsection{信用评价的分类}

信用评价的主要类型和分类可以根据不同的标准来划分。以下是几种常见的分类方式:

按评价对象分类:
企业信用评价:主要评估企业的偿债能力、履约能力、守信程度等,以判断企业的信用状况。
个人信用评价:主要评估个人的履约能力、信用记录、还款意愿等,以判断个人的信用状况。
国家信用评价:主要评估一个国家的偿债能力、政治稳定性、经济前景等,以判断国家的信用状况。

按评价范围分类:
综合信用评价:对企业或个人的整体信用状况进行全面评估,包括财务状况、经营能力、履约记录等多个方面。
专项信用评价:针对某一特定方面或领域进行信用评估,如债券信用评价、担保机构信用评价等。

按评价方法分类:
定性评价:主要依赖于评估人员的经验和专业知识,通过主观判断来评估信用状况。
定量评价:主要依赖于数学模型和统计数据,通过客观分析来评估信用状况。

按评价机构分类:
内部信用评价:由企业或机构内部设立的信用评价部门进行,主要用于内部管理和风险控制。
外部信用评价:由独立的第三方信用评价机构进行,结果更具公正性和权威性,广泛应用于市场。

按评价等级分类:
信用评价通常分为不同的等级,如AAA级、AA级、A级、BBB级、BB级、B级、CCC级、CC级、C级和D级等。不同机构对等级的定义可能略有差异,但一般来说,AAA级表示信用状况最好,违约风险最低;而D级则表示信用状况最差,违约风险最高。

本文研究的信用评价的类别是企业客户的信用等级,将企业客户的信用等级进行划分,为A公司信用相关人员提供一个清晰的信用参考,帮助他们做出更明智的决策,降低客户的信用风险,促进商业合作。 

%\subsection{是否逾期的定义}
%本文基于A公司的信用评价业务基础进行研究,该业务中,客户在申请信用评价后,会得出该客户的账期,如果客户在账期内未付清账款,发生逾期,逾期产生七天后仍逾期,则该客户的逾期记录被归档。

%客户历史是否有逾期的定义如下,若一个客户在申请信用评价时,若过去6个月以内存在逾期归档记录,则该客户就被记录为有逾期记录。本文延用该定义,在此基础上,研究客户未来是否会产生逾期。未来是否%%会逾期定义为:若一个客户在申请信用评价时,未来6个月以内会产生逾期且会被归档,则该客户未来是否会逾期将被记为是,否则为否。


\section{二元Logistic逻辑回归}
%\subsection{}
%Logistic回归它可以从多个自变量中选出对因变量有影响的自变量,并可以给出预测公式用于预测。Logistic回归分析分为3类,分别是二元Logistic回归分析、多元有序Logistic回归分析和多元无序Logisti%c回归分析。而因变量为二分类的称为二项logistic回归,通常再解释变量为0和1二值品质变量的时候采用。 

%Logistic二元逻辑回归是一种用于处理二分类问题的回归分析方法,是一个概率模型。它主要用于因变量为分类变量的回归分析,自变量可以为分类变量,也可以为连续变量,将自变量的线性组合映射到一个介%%于0和1之间的数值,表示某个样本属于某一类的概率。在实际应用中,可以通过设定一个阈值来将概率值转化为类别标签。逻辑回归的解释通常包括自变量的系数(即回归系数),这些系数可以用来解释自变量对%因变量的影响程度。逻辑回归模型还可以提供概率预测,帮助理解不同自变量对于预测结果的影响。


\subsection{sigmoid函数}
二元Logistic回归假设输出变量服从伯努利分布,即取值为0或1的离散分布。模型的目标是通过给定的输入变量,预测输出变量为0或1的概率。模型的核心是Logistic函数,该函数可以将输入变量的线性组合映射到一个0到1之间的数值。这种映射使得Logistic回归能够处理二分类问题。logistic函数也就是经常说的sigmoid函数,是一个S型曲线,公式如下:

\begin{equation}
	y = \frac{1}{1+\eu^-x} 
\end{equation}
sigmoid函数图像如下:

\begin{figure}[!h]
	\centering
	\includegraphics[width=.7\linewidth]{../../../../../pictures/logistic}
	 \caption{Sigmoid函数图像}
\end{figure}

因变量y为二分变量,其中阳性值取1,阴性值取0。假设p为取阳性的概率,那么1-p就是阴性的概率。阳性与阴性的比值就成为胜算(odds),定义胜算的公式为:

\begin{equation}
	odds = \frac{p}{1-p} 
\end{equation}


该公式表示阳性的概率是阴性概率的多少倍。对胜算取自然对数,就可以得到Logistic的逻辑回归模型。Logistic回归模型是建立,在$ln(\frac{p}{1-p})$与自变量的线性回归模型。公式如下:

\begin{equation}
	ln(\frac{p}{1-p}) = f(x) = \beta_0+\beta_1X_1+\beta_2X_2+\cdots+\beta_kX_k
\end{equation}

其中$ln(\frac{p}{1-p})$服从二元logistic分布,取值范围为(-$\infty$, +$\infty$), $X_1$到$X_k$是k个自变量,取值范围可以是任意范围。自变量可以是连续变量,也可以是分类变量。如果X是分类变量,则需要将其处理为虚拟变量。方程的右边是一个线性回归方程,左边为胜算odds自然对数,经过对数转换,即可得到Logistic回归模型的线性模型。其中系数$\beta$是回归系数值,即为二元logistic回归需要计算出来的值,X增加1个单位,则胜算的自然对数会增加$\beta$的倍数。
则取阳性的概率p可以表示为:

\begin{equation}
	p= \frac{1}{1+e^{-(\beta_0+\beta_1X_1+\beta_2X_2+\cdots+\beta_kX_k)}}
\end{equation}

\subsection{回归模型估计}
(1)模型的参数估计

Logistic二元回归是使用极大似然估计(MLE)做参数估计。极大似然估计的基本思想是在给定一组观测数据的情况下,找到一组参数,使得这组参数能够最大化产生观测数据的概率。在Logistic回归的上下文中,这意味着找到一组参数(通常是权重和偏置),使得根据这些参数计算出的观测对象属于某个类别的概率与实际观测到的标签最匹配。通过极大似然估计,可以得到Logistic回归模型的参数估计值,进而使用这些参数来预测新观测对象的类别概率。

(2)拟合优度的检验

评估Logistic回归模型拟合优度的统计检验方法为Hosmer-Lemeshow检验,Hosmer-Lemeshow检验的基本思想是将模型预测的概率分成若干组,然后比较每组的实际观测值和预期值之间的差异。具体来说,它将样本按照模型预测的概率从低到高排序,然后分为若干个等概率的组。在每个组内,计算实际观测事件发生的比例和模型预测的事件发生的比例,并计算它们的差异。如果模型拟合得好,那么这些差异应该很小。

Hosmer-Lemeshow检验的原始假设是测量值分布和期望值分布之间没有显著性差异。如果检验的显著性P结果大于显著水平(一般取0.05),则接受原假设,模型拟合优度较好;当P值小于显著性水平时,拒绝原假设,模型拟合度较差。

(3)模型的假设检验
Logistic回归模型的假设检验,常用的检验方法有似然比检验(likelihood ratio test)、计分检验(score test)和Wald检验(Wald test),这些检验方法用于检验模型中各个自变量的系数是否显著,即判断自变量对因变量的影响是否存在统计学意义。

likelihood ratio检验:用于比较一个完整模型和一个简化模型之间的拟合优度,从而判断模型是否需要包含特定的自变量。基本思想是比较两种不同假设条件下,对数似然函数值的差别大小。它可以用来检验整个模型是否具有统计学意义,即所有自变量的总体回归系数是否均为0。在应用LR检验时,需要满足适用数据类型、样本容量要求、独立性假设、线性关系假设、无多重共线性等条件。此外,还需要对模型的拟合度和预测能力进行评估,并进行结果的解释和推断。只有在满足这些条件的情况下,才能获得准确可靠的LR检验结果,并对研究问题进行科学合理的判断和决策。LR检验的基本假设之一是样本之间是相互独立的。即各个样本之间的观测值不会相互影响。此假设在实际研究中往往不容易满足,因此需要在样本选择和实验设计上进行合理的控制,以尽量减少样本之间的相关性。LR检验的另一个重要假设是自变量和因变量之间存在线性关系。简单来说,就是自变量的变化对因变量的影响是线性的,而不是非线性的。如果自变量和因变量之间存在非线性关系,需要进行适当的数据变换或者选择其他适用的统计方法。

计分检验也是一种常用的假设检验方法,其基本原理和似然比检验类似,都是通过比较不同模型的对数似然函数值来进行检验。计分检验可以用于检验模型的某些假设是否成立,例如检验自变量是否满足线性关系或检验模型是否满足比例性假设等。计分检验的基本步骤构建模型的得分函数,计算得分函数的期望值:构建检验统计量,最后进行假设检验:根据检验统计量的分布和给定的显著性水平,进行假设检验。如果检验统计量的值大于临界值,则拒绝原假设,认为模型的某些假设不成立;否则,接受原假设。

Wald检验被广泛应用于医学、社会科学、金融等领域中,用于判断实验结果是否具有统计显著性,从而推断出实验中所探究的变量之间是否存在相关性。由于Wald检验的计算方法比较易懂,所以在统计学中也被广泛使用。本文采用Wald检验,Wald检验是每个自变量是否与因变量显著性影响,如果有多重共线性时结果不准确。Wald检验是先对原方程(无约束模型)进行估计,得到参数的估计值,再代入约束条件检查约束条件是否成立。Wald检验的优点是只需估计无约束一个模型。用u检验或者x平方检验,推断各参数B是否为0,其中使用Wald统计量来检验每个参数的显著性。Wald统计量的计算公式为:

\begin{equation}
	W = \frac{b-\beta}{SE(b)} 
\end{equation}

其中,b为模型估计的系数,β为假设的值,SE(b)为b的标准误。计算出W值后,可以将其与标准正态分布进行比较,如果W值大于标准正态分布的临界值,则说明该参数显著。
需要注意的是,Wald检验通常适用于样本量较大的情况,如果样本量较小,则可能会出现W值过大而导致显著性判断错误的情况。


\section{BP神经网络}
\subsection{BP神经网络信号传播}
BP神经网络(Back Propagation),作为一种多层前馈神经网络,核心特性在于信号的正向传递与误差的反向传播。该网络由输入层、隐含层和输出层构成,每一层均含有多个并行工作的神经元。在正向传播阶段,输入信号通过输入层与隐含层,最终到达输出层。训练过程中,一旦输出值与预期值之间存在误差,该误差将通过反向传播的方式逐层传递,进而动态调整每个神经元间的阈值及其连接权值。经过多轮的训练迭代,模型的输出值会逐渐逼近预期值。以下是BP神经网络的简化结构图:

\begin{figure}[!h]
	\centering
	\includegraphics[width=.7\linewidth]{../../../../../pictures/bpnetwork.png}
	\caption{BP神经网络信号传播图}
\end{figure}

上图中,$x_1$到$x_n$是BP神经网络的输人值,$y_1$到$y_n$是BP神经网络的预测值,V和W为BP神经网络权值。从图可以看出,BP神经网络可以看成一个非线性函数,网络输人值和预测值分别为该函数的自变量和因变量。当输入节点数为、输出节点数为m时,BP神经网络就表达了从个自变量到m个因变量的函数映射关系。

\subsection{训练过程}
BP神经网络学习过程包括以下步骤,

(1)网络初始化:首先,需要对神经网络的各连接权值和阈值进行初始化,通常赋予它们(0,1)区间内的随机数值。同时,设定误差函数和计算精度值。确定输入层、输出层、隐含层节点个数,假设分别为n、m、M,其中M的公式一般取如下:

\begin{equation}
	M = \sqrt{n+m} + a 
\end{equation}

其中a的取值范围为1-10。BP神经网络对于分类问题一般使用交叉熵损失函数作为误差函数。
二分类时,交叉熵损失函数如下:

\begin{equation}
	L = -\left[y \log(p) + (1 - y) \log(1 - p)\right]
\end{equation}

其中,其中:y是实际标签(0或1)。p是模型预测为正类的概率。log 是自然对数。

多分类输出时,误差函数拓展如下:

\begin{equation}
L = -\sum_{i=1}^{C} y_i \log(p_i)
\end{equation}

其中,$C$ 是类别的总数,$y_i$ 是样本实际属于第 $i$ 个类别的指示函数($y_i = 1$ 如果样本属于第 $i$ 类,否则 $y_i = 0$),$p_i$ 是模型预测样本属于第 $i$ 个类别的概率。


(2)正向传播:在正向传播过程中,输入样本从输入层传入,经过各隐含层神经节点的处理(包括加权求和和非线性激活函数),然后逐层传递,最终从输出层输出信息。每个神经元接收来自前一层所有神经元的加权输入,然后加上一个偏置(Bias),并通过激活函数(Activation Function)来产生输出信号。

\iffalse
这个过程可以用以下数学公式表示:
对于隐藏层的神经元:

\begin{equation}
net_{j} = \sum_{i} w_{ij} x_{i} + b_{j}netj​=i∑​wij​xi​+bj​
out_{j} = f(net_{j})outj​=f(netj​)
\end{equation}

对于输出层的神经元:
\begin{equation}
net_{k} = sum_{j} w_{jk} out_{j} + b_{k}netk​=sumj​wjk​outj​+bk​
out_{k} = f(net_{k})outk​=f(netk​)
\end{equation}
其中 $w_{ij}wij​$ 是输入层第 ii 个神经元到隐藏层第 jj 个神经元的权重,$w_{jk}wjk​$是隐藏层第 jj 个神经元到输出层第 kk 个神经元的权重;$b_{j}bj​$ 和 $b_{k}bk​$ 分别是隐藏层和输出层的偏置;$f(cdot)f(cdot)$ 是激活函数,常用的激活函数包括 Sigmoid、Tanh、ReLU 等。
\fi

(3)计算输出层误差:计算网络的实际输出与期望输出之间的误差。这个误差反映了当前网络对输入样本的处理能力,也是后续权值调整的依据。

(4)误差反向传播:如果输出层的误差大于设定的精度值,那么将误差逐层反向传播到之前的各层。在这个过程中,根据误差信号调整各层神经元的权值和阈值,使网络的输出向期望输出逼近。

(5)权值更新:根据误差反向传播的结果,按照一定的学习率更新各层神经元的权值和阈值。这个过程是神经网络学习的核心,通过不断调整权值和阈值,使网络能够更好地适应输入样本。

(6)迭代学习:重复以上步骤,直到整个训练样本集的误差减小到符合要求为止。在这个过程中,神经网络的性能会逐渐提高,对输入样本的处理能力也会逐渐增强。

以上就是BP神经网络学习过程的主要步骤。需要注意的是,这个过程是一个迭代的过程,需要反复进行权值调整和误差计算,直到达到设定的精度要求或者达到预设的最大迭代次数。

\subsection{激活函数}
BP神经网络的激活函数(Activation Function)是在神经网络的神经元上运行的函数,负责将神经元的输入映射到输出端。激活函数对于神经网络模型去理解、学习复杂的非线性函数具有十分重要的作用。几种常用的激活函数如下:

(1) Sigmoid函数

Sigmoid函数已在二元Logistic回归中介绍过,见公式(2.1),BP神经网络和Logistic回归中使用的Sigmoid函数在本质上是一样的,都是S型的非线性函数,可以将连续的输入映射到0到1的输出范围内。然而,在BP神经网络和Logistic回归中,Sigmoid函数的应用方式有所不同。

在BP神经网络中,Sigmoid函数通常用作隐藏层和输出层的激活函数,用于将神经元的输出映射到0到1的范围内。这使得网络能够学习和模拟非线性函数,从而提高了网络的表达能力和泛化能力。在训练过程中,BP算法通过反向传播误差来更新网络的权重和偏置,使得网络的输出逐渐逼近期望的输出。而在Logistic回归中,Sigmoid函数则用作输出函数,将回归模型的输出映射到0到1的范围内,从而得到一个概率值。这个概率值表示了样本属于正类的概率。在训练过程中,Logistic回归使用最大似然估计来求解模型的参数,使得模型能够最小化预测错误。

(2)Tanh函数

Tanh函数是双曲正切(Hyperbolic Tangent)函数的缩写,是Sigmoid函数的一种变体,它将连续的输入映射到(-1,1)的输出范围内。与Sigmoid函数相比,Tanh函数在零点附近的导数更大,从而可能在一定程度上减轻梯度消失的问题。

(3)ReLU函数(Rectified Linear Unit)

ReLU函数是近年来非常受欢迎的激活函数之一。对于输入值小于0的情况,ReLU函数的输出为0;对于输入值大于0的情况,ReLU函数的输出等于输入值。ReLU函数具有简单的数学形式和计算效率高的特点,并且在一定程度上能够缓解梯度消失的问题。然而,它也可能导致神经元输出恒为0的问题。

(4)Softmax函数

Softmax函数通常用于多分类问题的输出层。本文的输出就涉及到多分类的问题,所以会用到该函数。它将一组输入映射到一个概率分布上,使得所有输出的概率之和为1。Softmax函数在BP神经网络中用于将神经元的输出转换为概率值。Softmax函数的定义如下:

\begin{equation}
	Softmax(z_i) = \frac{e^{z_i}}{\sum\limits_{k=1}^{K} e^{z_k}} 
\end{equation}

其中$z_i$为第i个节点的输出值,K为输出节点的个数,即分类的类别个数。通过Softmax函数就可以将多分类的输出值转换为范围在[0, 1]和为1的概率分布。
\subsection{梯度下降法}
梯度下降法是一种优化算法,用于寻找最小化损失函数的参数。BP神经网络中通过训练误差来逐步调整各层间的输入权重和偏置,这个调整的过程依据的算法一般有两种,一个是梯度下降法(Gradient Descent),另一个是最小二乘法。

训练误差(或称为损失函数)是依赖于神经网络中的权重和偏置的二次函数。为了最小化这个误差,需要计算损失函数关于权重和偏置的偏导数,这些偏导数组成了梯度向量。梯度向量的方向指示了训练误差增加最快的方向,而梯度向量的相反方向则是训练误差减少最快的方向。因此,沿着梯度向量的反方向进行权重和偏置的更新,可以更有效地找到损失函数的最小值,从而实现神经网络的优化。

\begin{figure}[!h]
	\centering
	\includegraphics[width=.7\linewidth]{../../../../../pictures/dituxiajiang.png}
	\caption{损失函数图}
\end{figure}

假设上图中描绘的曲线代表了损失函数的形状,如果设置的步长太小,迭代次数就会多,收敛慢,如果设置的步长太大,会引起震荡,可能导致无法收敛。该函数存在一个最小值点,是误差率最小,也就是斜率最小。梯度是指损失函数在某一点处的方向导数,沿着这个方向,损失函数的变化率最大。在BP神经网络中,梯度下降法通过计算损失函数关于权重和偏置的偏导数(即梯度),然后沿着梯度的反方向更新权重和偏置,以减小损失函数的值。这个过程不断迭代进行,直到损失函数的值收敛到一个最小值。这种方法就是通过不断地迭代更新参数,最终实现网络的优化。
假设损失函数为E,权重为w,偏置为b,学习率为η。在梯度下降法中,按照以下公式更新权重和偏置:

对于权重:

\begin{equation}
w_{new} = w_{old} - \eta \frac{\partial E}{\partial w}
\end{equation}

对于偏置:

\begin{equation}
b_{new} = b_{old} - \eta \frac{\partial E}{\partial b}
\end{equation}

其中,$(\frac{\partial E}{\partial w})$ 和 $(\frac{\partial E}{\partial b})$ 分别表示损失函数E对权重w和偏置b的偏导数,η是学习率,它决定了参数更新的步长。

这两个公式表示,在每次迭代中,都按照损失函数在当前点的梯度方向(即损失函数增加最快的方向)的相反方向来更新权重和偏置,从而逐渐逼近损失函数的最小值。

梯度下降法有多种变体,如批量梯度下降法、随机梯度下降法和小批量梯度下降法等。这些方法的主要区别在于每次更新权重和偏置时所使用的样本数量不同。批量梯度下降法使用所有的样本数据进行计算,而随机梯度下降法则每次只使用一个样本数据进行计算。小批量梯度下降法则是介于两者之间,每次使用一部分样本数据进行计算。不同的方法在不同的应用场景下可能有不同的表现,需要根据实际情况进行选择。
\subsection{模型检验}

如果要查看BP神经网络模型建立的效果如何,可以通过以下几个方面进行检验:

误差下降曲线图:BP神经网络预测模型结果需要查看误差下降曲线图。然而,一般不必过分关注这条曲线,除非是为了研究改进算法以提高收敛速度。相比之下,更应关注网络的实际训练效果和实际应用能力,例如预测能力等。

训练集和测试集的表现:可以通过观察模型在训练集和测试集上的表现来评估模型的效果。如果模型在训练集上表现良好,但在测试集上表现不佳,可能出现了过拟合的问题。相反,如果模型在训练集和测试集上的表现都不好,可能是模型复杂度不足,即欠拟合。

过拟合和欠拟合的解决方法:针对过拟合问题,可以尝试增加训练数据量、使用正则化约束、调整参数和超参数、降低模型复杂度或使用Dropout等方法。对于欠拟合问题,可以尝试增加模型的复杂度、增加样本有效特征数、调整参数和超参数或使用更复杂的网络结构等方法。

实际应用效果:最终,模型的效果还需要通过在实际应用中的表现来评估。例如,可以观察模型在预测任务中的准确率、召回率、F1分数等指标,以及模型在处理实际问题时的稳定性和鲁棒性。
综上所述,BP神经网络模型建立的效果需要综合考虑多个方面的因素来评估。在实际应用中,可能需要根据具体问题和数据集的特点来选择合适的评估方法和优化策略。
\section{本章小结}

本章探讨了客户信用评价的目的及分类标准,同时详细阐述了在模型构建过程中所使用的二元Logistic回归和BP神经网络的核心理论及技术点。通过对这些理论和技术点的细致剖析,旨在更深刻地理解它们在模型构建中所发挥的作用,从而为读者提供一个全面而深入的理论框架。

在二元Logistic回归部分,重点介绍了在筛选指标过程中所使用的函数、参数估计方法以及模型检测的相关技巧。这些函数和方法的运用,能够有效地识别并筛选出与信用评价密切相关的关键指标,进而构建出更加精准和可靠的预测模型。

在BP神经网络部分,详细阐述了信号传播的原理及过程,并介绍了在训练过程中所使用的步骤及函数。此外,还介绍了梯度下降法的损失函数相关理论,以及BP神经网络模型检验的方法。这些理论和方法的运用,有助于更好地理解和应用BP神经网络。