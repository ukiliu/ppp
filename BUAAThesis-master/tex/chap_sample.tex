\chapter{信用评价模型构建}

\section{基于二元Logistic回归构建指标体系}

\subsection{数据选取}
本研究的数据来源是从A公司的信用管理系统及ERP系统中获取到的数据,这些数据是关于企业客户在信用管理系统中做过信用评价的客户数据,部分数据是通过企查查第三方信息查询机构中获取的,比如企业是否上市、不良记录等信息。涉及了从2019年至2023期间A公司所有进行过信用评价的客户,考虑到国外客户的信息无法获取的因素,该二元Logistics研究只取了A公司国内客户的数据,排除无财务相关指标的数据及错误数据的客户,获取到的初始数据有1861条。其中下一节中介绍的指标均为模型中的自变量,本模型的因变量为是否逾期FutureOverdueFlag。是否逾期的定义为:客户在信用管理系统进行过信用评价的申请后,如果在6个月内存在应收账款逾期记录,则就被标记为逾期。

是否逾期说明:
本文基于A公司的信用评价业务基础进行研究,该业务中,客户在申请信用评价后,会得出该客户的账期,如果客户在账期内未付清账款,发生逾期,逾期产生七天后仍逾期,则该客户的逾期记录被归档。
客户是否有历史逾期的定义:若一个客户在申请信用评价时,若过去6个月以内存在逾期归档记录,则该客户就被记录为有逾期记录。本文延用该定义,在此基础上,研究客户未来是否会产生逾期。
未来是否会逾期定义为:若一个客户在申请信用评价时,未来6个月以内会产生逾期且会被归档,则该客户未来是否会逾期将被记为是,否则为否。

\subsection{指标筛选}
1.财务指标

在参考了国内外评级机构及信用评价相关文献中,本研究选择的财务指标最核心的四大指标,有盈利能力、变现能力、偿债能力、资产管理能力,这些是企业信用评价非常重要的财务指标,它们能够为信用评价时提供企业是否有全面的财务健康状况,企业在面对短期偿债压力时的应对能力,债务偿还能力以及风险应对情况的评估。指标详见表1:

%\begin{table}[!h]
%	\centering
%	\caption{信用评级财务指标}
%	\label{tab:exampletable}
%	\begin{tabular}{p{4cm}p{4cm}}
%		\toprule
%		\multicolumn{1}{c}{\textbf{一级指标}} & \multicolumn{1}{c}{\textbf{TeX 二级}} \\
%		\midrule
%		所有 & TeX Live \\
%		macOS & MacTeX \\
%		Windows & MikTeX \\
%		\bottomrule
%	\end{tabular}
%\end{table}

\begin{table}[h]
	\caption{信用评级财务指标}
	\label{tab:papercomponents}
	\centering
	\begin{tabular}{ccc}
		\toprule
		{\bfseries 一级指标} &  {\bfseries 二级指标} & {\bfseries 指标符号}  \\
		\midrule
		盈利能力 &  销售净利率 & NPM \\
		 & 销售毛利率          &  GPM\\
		 & 净资产收益率 & ROE\\
		 \midrule
		变现能力 & 流动比率        & CR\\
		 & 速动比率        & QR\\
		 & 现金比率      & CASHR\\
		  \midrule   
		资产管理能力 & 存货周转率  & ITR \\
		 & 应收账款周转率        & ARTR\\
		 \midrule
		偿债能力 & 资产负债率        & DAR\\
		\bottomrule
	\end{tabular}
\end{table}

2.业务指标

  业务指标是筛选出与A公司有业务往来的客户的客观数据,分为三个指标:信用记录、增信措施、经营指标。信用记录指标反映的是企业在过去的与A公司的经营活动中所形成的信用行为和信用历史,可以反映企业过去的信用表现,展现客户的信用行为和信用管理能力,信用记录良好的客户通常能够获得更好的信用评级。增信措施是客户为提高信用而采取的措施,例如提供担保、投保信用险等。增信措施可以帮助企业降低债权人的信用风险,增强债务偿还能力。经营指标是企业的各项经营数据和,这些指标可以反映企业的经营状况。良好的经营指标通常可以为企业赢得更好的信用。本文采用的业务指标主要有以下指标:
  
  \begin{table}[h]
	\caption{信用评级业务指标}
	\label{tab:papercomponents}
	\centering
	\begin{tabular}{ccc}
		\toprule
		{\bfseries 一级指标} & \multicolumn{1}{c} {\bfseries 二级指标} & \multicolumn{1}{c} {\bfseries 指标符号}  \\
		\midrule
		信用记录 &  是否有逾期记录 & OverdueFlag \\
		& 不良记录          &  BADR\\
		\midrule
		增信措施& 是否有银行保函 & BGFlag\\
		& 是否投保信用险        & CTFlag\\
		& 是否关联担保公司        & GCFlag\\
		\midrule
		经营指标& 客户经营状态            & CES\\
		& 经营负荷情况  & OLC \\
		& 内销比例        & DSR\\
		& 外销比例        & ESR\\
		& 是否为新客户        & NewFlag\\
		\bottomrule
	\end{tabular}
\end{table}


3.企业资质指标

企业资质指标可以反映企业的经营实力和规模水平,也可以反映企业的历史背景和稳定性,也能展示出企业的发展潜力和未来增长空间。拥有相关的资质认定可以表明企业在特定领域具有一定的专业能力和技术实力,同时也有助于提升企业的品牌形象和知名度,这些对于企业的信用评价具有积极的影响。企业成立年限是衡量企业稳定性和可靠性的重要指标之一,成立年限较长的企业在面对市场波动和风险挑战时,往往具有更强的抵抗能力和适应能力,其信用状况也更为可靠。获得ISO认证意味着企业已经通过了一系列国际标准的审核,证明其在质量、环境、安全等方面达到了一定的要求。在信用评价中,拥有ISO资格认证的企业往往更容易获得较高的信用评级。上市公司需要遵守更为严格的法规和监管要求,包括定期公开财务报告、接受审计和遵守证券市场的相关规定。这种透明度和规范性有助于提升公司的信用评级。净资产代表企业所有者对企业资产的净权益,它反映了企业的自有资金实力。一个拥有较大净资产规模的企业,通常意味着其具备较强的财务实力和抵御风险的能力。在信用评价中,这种财务实力是企业获得较高信用评级的重要保障。
本文采用的企业资质指标如下表:

  \begin{table}[h]
	\caption{信用评级企业资质指标}
	\label{tab:papercomponents}
	\centering
	\begin{tabular}{ccc}
		\toprule
		{\bfseries 一级指标} & \multicolumn{1}{c} {\bfseries 二级指标} & \multicolumn{1}{c} {\bfseries 指标符号}  \\
		\midrule
			基本信息 &  企业成立年限 & YOE \\
		& ISO资格认证         &  ISOFlag\\
		& 是否上市 & ListedFlag \\
		& 净资产规模        & NAS\\
		\bottomrule
	\end{tabular}
\end{table}


\subsection{数据预处理}
在做二元Logistic回归分析之前,通常需要先进行一系列的数据分析工作。这些分析不仅有助于确保数据的准确性和可靠性,还能为后续的Logistic回归分析提供重要的背景信息和指导。本文将对企业数据进行深入的相关性分析和多重共线性分析,旨在揭示企业不同变量之间的关联程度以及变量间可能存在的潜在问题。

通过相关性分析,可以了检测探究不同变量之间是否存在某种关联,以及这种关联的程度和方向。这对于理解研究领域的内在机制和影响因素至关重要,为后续模型解释提供基础。同时相关性分析是进行多因素 Logistic 回归分析的基础。在进行多因素分析之前,需要首先了解各自变量与因变量之间的相关性,以便选择最具影响力的自变量进行进一步的分析。同时,多重共线性分析将帮助处理变量间的共线性问题,避免自变量之间存在高度相关的情况,当存在多重共线性时,回归模型的估计结果可能会失真,导致模型对数据的拟合效果不佳,降低模型的稳定性和可靠性。通过综合运用这两种分析方法,将更全面地了解数据的内在关系,为构建准确、可靠的模型打下坚实的基础。

Logistic回归模型本身是一种概率模型,它分析的是某一事件发生与否的概率与自变量之间的关系。因此,即使变量之间存在高度相关性,Logistic回归模型仍然可以处理并给出相应的结果。但是,为了获得更稳定、更准确的模型,需要在进行Logistic单因素分析之前先进行相关性分析。

(1) Pearson相关系数分析

本文采用Pearson相关系数分析,计算公式为

\begin{equation}
	r = \frac{cov(X, Y)}{\sigma X * \sigma Y} 
\end{equation}

其中cov(X, Y)表示X和Y的协方差,$\sigma$X和$\sigma$Y表示X和Y的标准差。相关系数的取值范围在-1到1之间,当r>0时表示正相关,r<0时表示负相关,r=0时表示无相关关系。

使用Excel数据分析工具对3.1.2章节得出的23个指标数据进行分析得出下表:

\begin{table}[h]
	\caption{指标相关系数r}
	\label{tab:papercomponents}
	\centering
	\begin{tabular}{cccc}
		\toprule
		相关系数(r) & 流动比率(CR) & 速动比率(QR) &  内销比例(DSR) \\
		\midrule
		速动比率(QR) &  0.98  &  1 &  0.088 \\
		现金比率(CASHR) &  0.86  &  0.88 &  0.1 \\	
		外销比例(ESR)&  -0.088  &  -0.088 &  -1 \\	
		\bottomrule
	\end{tabular}
\end{table}

	经过深入的相关性分析,发现速动比率(QR)与流动比率(CR)的相关系数高达0.98,现金比率(CASHR)与流动比率(CR)的相关系数为0.86,而现金比率(CASHR)与速动比率(QR)的相关系数也达到了0.88,均显示出强烈的正相关性。同时,外销比例(ESR)与内销比率(DSR)之间的相关系数为-1,呈现出极强的负相关性。
	
	鉴于速动比率(QR)与流动比率(CR)以及现金比率(CASHR)之间的强相关性,优先选择了保留现金比率(CASHR),因为它直接反映了企业可立即用于偿还短期债务的能力,是信用评价中的关键指标。较高的现金比率意味着企业短期偿债能力更强,从而有助于提升信用评级。
	
	在内销比例(DSR)和外销比例(ESR)的选择上,保留了外销比例(ESR)作为代表性指标。最终,剔除了速动比率(QR)、流动比率(CR)和内销比率(DSR)。
	
	此外,在相关性分析中,注意到指标“是否有银行保函”(BGFlag)在排除无财务相关指标的数据后,该指标的数据均变为无银行保函,因此不再具有研究价值,删除该指标。
	
	经过这一系列的分析与筛选,最终保留了19个二级指标,这些指标将为后续的研究提供更加精准和有效的数据支持,如下表:

\begin{table}[h]
	\caption{信用评级指标}
	\label{tab:papercomponents}
	\centering
	\begin{tabular}{cccc}
		\toprule
		{\bfseries 指标大类} & {\bfseries 一级指标} &  {\bfseries 二级指标} & {\bfseries 指标符号}  \\
		\midrule
		财务指标 & 盈利能力 &  销售净利率 & NPM \\
		& & 销售毛利率          &  GPM\\
		& & 净资产收益率 & ROE\\
		\midrule
		& 变现能力 & 现金比率      & CASHR\\
		\midrule   
		& 资产管理能力 & 存货周转率  & ITR \\
		&	& 应收账款周转率        & ARTR\\
		\midrule
		&偿债能力 & 资产负债率        & DAR\\
		\midrule
		& 信用记录 &  是否有逾期记录 & OverdueFlag \\
		& 不良记录          &  BADR\\
		\midrule
		业务指标&增信措施& 是否投保信用险        & CTFlag\\
		& & 是否关联担保公司        & GCFlag\\
		\midrule
		& 经营指标& 客户经营状态            & CES\\
		& & 经营负荷情况  & OLC \\
	   &	& 外销比例        & ESR\\
		& & 是否为新客户        & NewFlag\\
		\midrule
		企业资质&基本信息 &  企业成立年限 & YOE \\
		& & ISO资格认证         &  ISOFlag\\
		& & 是否上市 & ListedFlag \\
	& 	& 净资产规模        & NAS\\
		\bottomrule
	\end{tabular}
\end{table}
(2)多重共线性分析

为了避免自变量之间存在高度相关性的情况,需要对变量进行多重共线性检测。多重共线性可能会导致回归系数估计不准确,使得模型的解释性和预测能力降低。本文采用方差膨胀因子VIF进行检查,计算公式如下, 

\begin{equation}
	VIF =1/(1-R^2) 
\end{equation}

$R^2$为该自变量与其他自变量的线性相关系数的平方和。

本论文对保留的19个指标进行了共线性检测。计算各指标之间的方差膨胀因子(VIF),检测结果如下:
\begin{table}[h]
	\caption{指标多重共线性分析}
	\label{tab:papercomponents}
	\centering
	\begin{tabular}{cc}
		\toprule
		指标 & VIF \\
		\midrule
		是否为新客户NewFlag	&	1.447	\\
		企业成立年限YOE	&	1.592	\\
		ISO资格认证ISO	&	1.112	\\
		是否有逾期记录OverdueFlag	&	1.152	\\
		其他不良记录BADR	&	1.348	\\
		是否上市ListedFlag	&	1.446	\\
		经营负荷情况OLC	&	1.099	\\
		是否已投保信用险CTFlag	&	1.216	\\
		是否关联担保公司CTFlag	&	1.212	\\
		销售净利率NPM	&	1.352	\\
		销售毛利率GPM	&	1.574	\\
		净资产收益率ROE	&	1.308	\\
		资产负债率DAR	&	2.347	\\
		净资产规模(人民币万元)NAS	&	1.482	\\
		现金比率CASHR	&	1.905	\\
		存货周转率ITR	&	1.171	\\
		应收账款周转率ARTR	&	1.048	\\
		外销比例ESR	&	1.242	\\
		\bottomrule
	\end{tabular}
\end{table}

当0<VIF<10,不存在多重共线性;当10≤VIF<100,存在较强的多重共线性;当VIF≥100,存在严重多重共线性。在上述表中,VIF值最大的为2.347<3,发现这些指标之间不存在严重的共线性问题。这意味着这19个指标能够独立地为本研究提供有价值的信息,避免了因共线性导致的分析结果失真。

\subsection{指标数据的处理}
数据归一化是将数据归一化至[0,1]区间,消除不同变量之间性质、量纲、数量级等特征属性的差异,将其转化为一个无量纲的相对数值,也就是标准化数值,使各指标的数值都处于同一个数量级别上,从而便于不同单位或数量级的指标能够进行综合分析和比较。数据标准化在模型运行之前具有重要意义,可以提高模型的稳定性、准确性和收敛速度。数据标准化有以下重要功能:

(1)统一量纲:数据标准化可以将原始数据的量纲转化为无量纲的纯数值,便于不同单位或量级的指标能够进行比较和加权。这样做可以消除由于量纲不同导致的误差,使得模型能够更好地获取数据中的有效特征。

(2)消除异常数据的影响:标准化能够减小异常数据对于模型训练的影响,从而加快模型的收敛速度。标准化处理可以消除奇异数据对模型训练的负面影响,提高模型的稳定性和准确性。

本文19种指标中有不同的数据类型,有分类变量和连续变量。分以下步骤进行数据的标准化:

(1)数据清洗,为减少数据冗余提高数据质量和准确率,本文对原始数据进行了删除重复项、删除异常值,删除了缺少财务相关指标数据的处理后,剩余239条数据。

(2)数据转换和规范化,其中企业成立年限需要根据企业的注册日期计算出相应的年限;对于分类变量客户经营状态、ISO资格认证、是否有逾期记录、其他不良记录、是否上市、经营负荷情况、是否已投保信用险、是否关联担保公司的指标数据规范成为0和1;是否为新客户,新客户值为1,老客户值为0;因变量是否逾期也规范为0和1,其中0表示否定,1表示肯定。

本研究将企业成立年限作为分类变量,根据成立年限的长短,将企业成立年限划分为不同的级别,每一级别给定一个具体的分数范围,以便进行量化考核。以下表格是企业成立年限处理后的结果。

\begin{table}[h]
	\caption{企业成立年限的标准化处理}
	\label{tab:papercomponents}
	\centering
	\begin{tabular}{ccc}
		\toprule
		{\bfseries 指标名称} & \multicolumn{1}{c} {\bfseries 评分标准} & \multicolumn{1}{c} {\bfseries 分值}  \\
		\midrule
		企业成立年限YOE &  y≥10年 & 1.0 \\
		            & 5年≤y<10年      &  0.7\\
		            & 3年≤y<5年      &  0.4\\
		            & 1年≤y<3年      &  0.2\\
		            & y<1年      &  0\\
		\bottomrule
	\end{tabular}
\end{table}

(3)对连续变量数据的归一化。

对其他连续变量本文采用最小-最大归一化Min-Max Normalization标准化,是一种常用的数据预处理技术,用于将数据缩放到一个特定的范围,通常是0到1之间。这种方法对于消除数据中的量纲和量纲单位的影响非常有用,使得不同特征或指标之间可以进行比较和加权。最小-最大归一化的数学公式如下:

\begin{equation}
	x^{'} = \frac{x - min(x)}{max(x)-min(x)} 
\end{equation}

其中,x 是原始数据集中的某个特征值,min(x) 是该特征在所有数据中的最小值,max(x) 是该特征在所有数据中的最大值,$x^{'}$是归一化后的值,它将落在0到1的范围内。

\subsection{单因素Logistic分析}
单独考察每一个自变量对因变量的净效应,有助于更深入地理解每一个因素在整体模型中的作用和贡献。为了检测单一因素对是否逾期这一因变量的影响,判断这些自变量和因变量之间的关系是否显著,了解自变量如何影响因变量的概率,保留显著的自变量。单因素的数学模型构建如下:


\begin{equation}
	ln(\frac{p}{1-p}) = \beta_0+ \beta_i X_i
\end{equation}

本公式为公式2.3变量个数为1的情况,$\beta_0$为常量,$X_i$是自变量,$\beta_i$是自变量的系数。

对标准化后的19个自变量逐一与是否逾期进行二元Logistic逻辑回归,利用SPSS工具进行回归得到的显著性P值表如下:

\begin{table}[h]
	\caption{客户逾期单因素回归结果}
	\label{tab:papercomponents}
	\centering
	\scalebox{0.7}{
	\begin{tabular}{ccccccccc}
		\toprule
		 & \multirow{2}{*}{B} & \multirow{2}{*}{标准误差}&\multirow{2}{*}{瓦尔德}	&\multirow{2}{*}{自由度}&	\multirow{2}{*}{显著性}&\multirow{2}{*}{Exp(B)}&  \multicolumn{2}{c}{EXP(B)的95\%置信区间 } \\
		 \cline{8-9}  
		 & & & & & & & 下限 & 上限\\ 
		 
		\midrule
		是否关联担保公司	&	1.419	&	0.726	&	3.817	&	1	&	0.051	&	4.133	&	0.995	&	17.162	\\
		经营负荷情况	&	-1.454	&	0.602	&	5.838	&	1	&	0.016	&	0.234	&	0.072	&	0.76	\\
		是否上市	&	-1.377	&	0.497	&	7.668	&	1	&	0.006	&	0.252	&	0.095	&	0.669	\\
		是否有逾期记录	&	0.959	&	0.436	&	4.835	&	1	&	0.028	&	2.609	&	1.11	&	6.135	\\
		应收账款周转率	&	-3.18	&	2.316	&	1.885	&	1	&	0.17	&	0.042	&	0	&	3.894	\\
		客户经营状态	&	-1.454	&	0.602	&	5.838	&	1	&	0.016	&	0.234	&	0.072	&	0.76	\\
		资产负债率	&	1.734	&	0.718	&	5.833	&	1	&	0.016	&	5.663	&	1.387	&	23.132	\\
		销售毛利率	&	-1.568	&	1.199	&	1.709	&	1	&	0.191	&	0.208	&	0.02	&	2.187	\\
		现金比率	&	-0.78	&	0.337	&	5.348	&	1	&	0.021	&	0.458	&	0.237	&	0.888	\\
		销售净利率	&	-4.547	&	2.485	&	3.348	&	1	&	0.067	&	0.011	&	0	&	1.382	\\
		ISO资格认证	&	-0.292	&	0.407	&	0.515	&	1	&	0.473	&	0.747	&	0.336	&	1.658	\\
		净资产收益率	&	-0.192	&	0.711	&	0.073	&	1	&	0.788	&	0.826	&	0.205	&	3.328	\\
		是否已投保信用险	&	-0.387	&	0.569	&	0.464	&	1	&	0.496	&	0.679	&	0.223	&	2.07	\\
		企业成立年限	&	-0.083	&	0.646	&	0.016	&	1	&	0.898	&	0.921	&	0.259	&	3.268	\\
		净资产规模(人民币万元)	&	0	&	0	&	0.066	&	1	&	0.797	&	1	&	1	&	1	\\
		ISO资格认证	&	-0.292	&	0.407	&	0.515	&	1	&	0.473	&	0.747	&	0.336	&	1.658	\\
		存货周转率	&	-0.018	&	0.05	&	0.133	&	1	&	0.715	&	0.982	&	0.891	&	1.083	\\
		其他不良记录	&	0.267	&	0.792	&	0.113	&	1	&	0.736	&	1.306	&	0.277	&	6.162	\\
		外销比例	&	-0.356	&	0.507	&	0.493	&	1	&	0.483	&	0.701	&	0.259	&	1.893	\\
		
		\bottomrule
	\end{tabular}
}
\end{table}

在统计学中,通常使用P值(P-value)来衡量某一变量或因素对研究结果的影响是否显著。Logistic二元逻辑回归的显著性水平通常是通过统计检验来确定的,例如使用p值来判断某个自变量是否对因变量有显著影响。显著性水平是一个预设的阈值,用于决定是否拒绝原假设(即认为自变量对因变量没有影响)。常见的显著性水平有0.05、0.01等,这意味着如果p值小于这些阈值,就拒绝原假设,认为自变量对因变量有显著影响。

由于本次研究所涉及的数据量相对较小,而且单因素分析也是初步筛选指标,为了充分利用有限的数据资源并尽可能保留与因变量“是否逾期”相关的指标,决定将显著性水平扩展到0.3。若变量的p值<=0.3则就被纳入下一步骤的多因素分析中。之所以设置p=0.3作为筛选标准,是考虑到数据量较少,避免遗漏某些重要的自变量\upcite{CJYJ200409002} 。观察上述表格的数据,可以发现是否关联担保公司、经营负荷情况、是否上市、是否有逾期记录、应收账款周转率、客户经营状态、资产负债率、销售毛利率、现金比率、以及销售净利率这十个自变量在统计检验中的显著性水平均低于0.3。这意味着这些变量在模型中具有较高的解释力,能够有效地反映出因变量的变化。因此,将这十个指标纳入Logistic多因素回归模型中,不仅是统计上的合理选择,更是为了更全面、更准确地揭示各因素之间的复杂关系及其对因变量的综合影响。

\subsection{多因素Logistic分析}
通过单因素分析,筛选出了10个与因变量是否逾期FutureOverdueFlag显著性较高的自变量。本文将充分利用这10个经过初步筛选和显著性水平放宽后保留的重要指标,进行多因素的二元Logistic回归分析。多因素二元Logistic逻辑回归的数学模型为公式2.3。利用SPSS工具,对数据进行了多因素的二元Logistic回归分析,采用了向后LR(向后似然比)的方法,设定步进概率为0.3,即当某个自变量的显著性水平大于0.3时,该自变量将被从模型中移除。模型最终迭代了4次后收敛,得到了包含6个自变量的模型,详情如下:
\begin{table}[h]
	\caption{客户逾期多因素回归结果}
	
	\label{tab:papercomponents}
	\centering
	\scalebox{0.8}{
	\begin{tabular}{ccccccccc}
		\toprule
		 & \multirow{2}{*}{B} & \multirow{2}{*}{标准误差}&\multirow{2}{*}{瓦尔德}	&\multirow{2}{*}{自由度}&	\multirow{2}{*}{显著性}&\multirow{2}{*}{Exp(B)}&  \multicolumn{2}{c}{EXP(B)的95\%置信区间 } \\
		\cline{8-9}  
		& & & & & & & 下限 & 上限\\ 
		\midrule
		经营负荷情况	&	-1.384	&	0.625	&	4.909	&	1	&	0.027	&	0.25	&	0.074	&	0.852	\\
		现金比率	&	-0.557	&	0.299	&	3.47	&	1	&	0.062	&	0.573	&	0.319	&	1.029	\\
		应收账款周转率	&	-3.792	&	2.903	&	1.706	&	1	&	0.191	&	0.023	&	0	&	6.67	\\
		是否有逾期记录	&	0.61	&	0.459	&	1.766	&	1	&	0.184	&	1.841	&	0.748	&	4.528	\\
		是否上市	&	-1.404	&	0.521	&	7.276	&	1	&	0.007	&	0.246	&	0.089	&	0.681	\\
		销售净利率	&	-2.94	&	2.639	&	1.241	&	1	&	0.265	&	0.053	&	0	&	9.329	\\
		
		\bottomrule
	\end{tabular}
}
\end{table}

多因素分析继续使用显著性<0.3作为检测标准,筛选了自变量,以确保所选指标对因变量“是否逾期”具有一定的解释力。在分析后得出的上表中,经营负荷情况、现金比率、应收账款周转率、是否有逾期记录、是否上市以及销售净利率这六个指标的显著性均小于0.3。

这些指标的显著性水平低于设定的阈值,表明它们在模型中起到了重要的作用,能够显著影响因变量“是否逾期”的变化。经营负荷情况反映了企业运营的压力和效率,其显著性说明了经营状况对逾期风险的影响;现金比率则体现了企业的流动性状况,对逾期风险具有直接的指示作用;应收账款周转率则揭示了企业应收账款的回收速度,与逾期风险密切相关。

此外,是否有逾期记录作为衡量企业信用状况的重要指标,其显著性小于0.3,进一步验证了信用历史对逾期风险的影响;是否上市则反映了企业的资本结构和市场监管情况,对逾期风险也有一定的影响;销售净利率则体现了企业的盈利能力和经营效率,对逾期风险同样具有解释力。

综上所述,这六个指标的显著性均小于0.3,基于这些重要指标,成功构建了一个回归模型。该模型不仅准确地反映了这些自变量与因变量之间的关系,还为后续的风险评估和决策制定提供了有力的数据支持。构建的回归模型如下:
\begin{equation}
	\begin{split}
	ln(\frac{p}{1-p}) &= 0.913-1.138OLC-0.557CASHR-3.792ARTR \\
	&+0.61OverdueFlag-1.404ListedFlag-2.94NPM
\end{split}
\end{equation}



\subsection{指标体系合理性检验}
1.完备性检验

  筛选出的6个指标经营负荷情况、现金比率、应收账款周转率、是否有逾期记录、是否上市以及销售净利率,在构建逾期风险模型时,涵盖了逾期风险的主要方面:
  
  经营负荷情况:这一指标反映了企业的运营压力和效率。过高的经营负荷可能意味着企业面临较大的运营风险,从而增加逾期风险。因此,经营负荷情况是评估企业逾期风险的重要因素之一。
  
  现金比率:现金比率是企业现金及现金等价物与流动负债的比值,反映了企业的短期偿债能力。现金比率较低的企业可能面临现金流紧张的问题,从而增加逾期风险。因此,现金比率是衡量企业逾期风险的重要指标。
  
  应收账款周转率:应收账款周转率反映了企业应收账款的回收速度。周转率较低可能意味着企业应收账款回收困难,导致现金流紧张,进而增加逾期风险。因此,应收账款周转率也是评估企业逾期风险的重要方面。
  
  是否有逾期记录:这一指标直接反映了企业的信用状况。有逾期记录的企业在未来发生逾期的可能性相对较高。因此,是否有逾期记录是判断企业逾期风险的重要依据。
  
  是否上市:上市与否反映了企业的资本结构和市场监管情况。上市公司通常受到更严格的监管和信息披露要求,这有助于降低逾期风险。因此,是否上市也是评估企业逾期风险的一个方面。
  
  销售净利率:销售净利率反映了企业的盈利能力和经营效率。净利率较低的企业可能面临较大的经营压力,从而增加逾期风险。因此,销售净利率也是衡量企业逾期风险的重要指标之一。

总之,选出的此6个指标能够很好地解释逾期风险。这些指标从不同维度反映了企业的运营状况、财务状况、信用状况以及市场状况,从而全面评估了企业的逾期风险。
2.拟合程度

基于3.1.6构建的二元Logistic回归模型,该模型的Hosmer-Lemeshow检验显著性如下表:
\begin{table}[h]
	\caption{霍斯默-莱梅肖检验}
	\label{tab:papercomponents}
	\centering
	\begin{tabular}{ccc}
		\toprule
		卡方 & 自由度 & 显著性 \\
		\midrule
		10.287 &  8  &  0.245 \\
		\bottomrule
	\end{tabular}
\end{table}


Hosmer-Lemeshow检验的显著性为0.245表明该检验的p值(概率值)为0.245,在统计学中,通常将p值与一个预定的显著性水平(如0.05)进行比较,以判断模型拟合优度的好坏。如果p值大于显著性水平,则通常认为模型的拟合程度是可以接受的,即模型预测的结果与实际观测结果之间的差异并不显著。本模型的p值为0.245,模型的拟合程度较好。
3.模型的准确率
根据表可以看到:
\begin{table}[h]
	\caption{分类表}
	\label{tab:papercomponents}
	\centering
	\begin{tabular}{ccccc}
		\toprule   
		\multirow{3}{*}{是否逾期}& &\multicolumn{3}{c}{预测} \\
		& & \multicolumn{2}{c}{预测}&正确百分比 \\
		& & 0&1& \\
		\midrule
		\multirow{2}{*}{是否逾期} &  0  &  178&12&93.6 \\
		 &  1  &  6&43&87.7 \\
		总体百分比  &     &   & &92.4 \\
		\bottomrule
	\end{tabular}
\end{table}

模型的准确率为92.4\%,有较强的预测能力。在预测客户未来信用情况时,该模型的结果与客户的信用等级相结合,将提供更具参考价值的洞察。因此,此模型在预测A公司企业客户未来是否会发生逾期行为方面,具有较高的实用性和准确性。

\section{基于BP神经网络构建评价模型}

\subsection{模型参数说明}
在指标构建章节的基础上,已经筛选了六个关键指标,这些指标对客户未来是否会发生逾期行为具有显著影响。本章节将利用这六个指标,构建一套全面而精准的信用评价模型。该模型不仅能够生成客户的信用等级,还能预测客户未来是否存在逾期风险。之所以同时提供这两个结果,是因为即使客户的信用等级很高,也不能百分之百地保证他们未来不会逾期。信用评价等级主要是基于客户过去的信用记录和行为进行预测和评估的,然而,过去的记录并不能完全代表未来的行为。因此,在给出信用等级的同时,还提供了未来逾期风险的判断,旨在为A公司在设定客户信用额度和制定账期时提供更为全面、科学的参考依据。这样,A公司可以更加精准地管理信用风险,优化资源配置,从而实现更为稳健和可持续的发展。输入输出字段详见表12:

\begin{table}[h]
	\caption{BP神经网络输入输出参数}
	\label{tab:papercomponents}
	\centering
	\begin{tabular}{cccc}
		\toprule
		{\bfseries 类型} & {\bfseries 参数}& {\bfseries 符号} & {\bfseries 参数类型}  \\
		\midrule
		输入 & 现金比率  & CASHR  &连续变量\\
		& 应收账款周转率 &ARTR&连续变量   \\
		&  经营负荷情况 &OLC& 二分类 0-否 1-是\\
		& 是否有逾期记录&OverdueFlag  & 二分类 0-否 1-是   \\		 
		& 是否上市   &ListedFlag& 二分类 0-否 1-是  \\
		& 销售净利率& NPM& 连续变量\\		 
		输出& 未来是否逾期 &FutureOverdueFlag & 二分类 0-否 1-是  \\
		& 信用等级& CreditRating & 多分类 值为1-5  \\
		\bottomrule
	\end{tabular}
\end{table}

在此BP神经网络模型中,输入层选取了六个关键指标,这些指标旨在全面捕捉客户信用的多个维度。而输出层,则聚焦于两个核心变量:“是否逾期”和“信用等级”。这两个变量对于评估客户的信用风险具有至关重要的作用。



“信用等级”这一关键输出,因为目前无法在各大评级机构中获取到A公司客户的信用等级。A公司已经建立了一套完善的客户等级评价体系,该体系在每次信用申请时都会为客户进行详尽的测评。这套评价系统是由大公国际信用评级集团有限公司倾力打造的,确保了评价标准的专业性和权威性。从而得出具有参考价值的信用等级。

同时,在评级过程中,A公司的多个关键部门,包括财务部门、信用管理部门以及法务部门等,都会进行逐层审核。这种跨部门的合作与审核机制,进一步提升了信用评级的准确性和可信度。因此,将这些经过严格审核的信用等级作为BP神经网络输出层的一部分,将为此模型提供强有力的支持,使其能够更加精准地预测客户的信用风险。

\subsection{模型建立}
BP神经网络模型BP网络模型包括其输入输出模型、作用函数模型、误差计算模型和自学习模型。
1.模型网络结构设置

基于BP神经网络的信用评价构建流程图如下:

\begin{figure}[!h]
	\centering
	\includegraphics[width=.7\linewidth]{../../../../../pictures/bpprocess.png}
	\caption{BP神经网络构建过程图}
\end{figure} 

BP神经网络是一种强大的多层前馈网络,其结构主要由输入层、隐藏层和输出层构成。每个神经元都与前一层的所有神经元通过权重连接,形成了密集的全连接模式。这种结构使得BP神经网络能够捕捉和表达复杂的非线性关系。

在构建BP神经网络时,网络的结构设计是至关重要的。它需要根据系统的输入输出数据特点来确定。具体而言,输入层的节点数应当与输入特征的维度完全匹配,以确保每一个特征都能被网络有效地捕获。对于输出层,其节点数则通常根据问题的性质来设定:在二分类问题中,输出层通常有一个节点;而在多分类问题中,则会有多个节点。

隐藏层,其节点数和层数的选择则相对灵活,往往需要根据实际情况进行调整。隐藏层的存在为网络提供了“深度”,使其能够学习并表达更复杂的模式和关系。通过适当地增加隐藏层的节点数或层数,网络可以更好地拟合数据,但也可能导致过拟合的风险增加。因此,在选择隐藏层的结构时,需要权衡好模型的复杂度和泛化能力\upcite{1013010799.nh}。

在本模型中,输入层有6个指标,因此输入层应设置6个节点。而输出层由于有两个指标需要预测,故应设置2个节点。对于隐藏层,可以参考一些经验公式或理论来设定其节点数。根据公式2.6,设定隐藏层的节点数为10。因此,本模型的BP神经网络结构最终确定为6-10-2。

假设中$x_1$,$x_2$,...,$x_n$是输入值,输出值为$y_1$,$y_2$,...,$y_m$,$\omega_{ij}$和 $a_j$为输入层与隐含层之间的权值和阈值,$\omega_{jk}$和b是隐含层与输出层间的权值和阈值。以下是各层输入输出的计算公式。

隐藏层的输出值计算公式如3.6。

\begin{equation}
	H_j = f(\sum\limits_{i=1}^{n}x_i\omega_{ij}-a_j) \quad  j=1,2,...,r
\end{equation}

其$H_j$为隐含层输出值,f是传递函数,r是隐含层节点数,在该模型中的值为10。

输出层输出值计算,根据隐含层的输出值H,连接权值$\omega_{jk}$和阈值$b_j$,计算出输出公式如下;

\begin{equation}
	O_k = f(\sum\limits_{i=1}^{r}H_j\omega_{jk}-b_k) \quad k=1,2,...,m
\end{equation}

其中$O_k$是输出层输出值,m为输出层的节点个数,即2。


%\begin{figure}[!h]
%	\centering
%	\includegraphics[width=.7\linewidth]{../../../../../pictures/bpmodel.png}
%	\caption{BP神经网络模型结构图}
%\end{figure}
%\clearpage  

2.函数选取

图6是网络在训练过程中所需要计算的流程图,BP神经网络在构建以及训练过程中,涉及多个关键的函数数,这些函数的选择会影响到模型建设效果。

 \begin{figure}
	\centering
	\includegraphics[width=.7\linewidth]{../../../../../pictures/bptrain.png}
	\caption{BP神经网络训练流程图}
\end{figure}


BP神经网络在构建过程中,涉及多种函数,主要涉及激活函数、学习函数、性能函数、初始化函数、传递函数。这些函数在神经网络的构建、训练和性能评估过程中起着关键作用。

激活函数是神经网络中非常关键的一个组件,它决定了神经元如何对其输入进行非线性变换。因为本研究的输出层两个指标,一个指标为是否逾期,该指标作为输出层是为二分类变量,选取Sigmoid函数,公式见(2.1),该函数常用于二分类问题的输出层,将数据映射到0和1之间;另一个指标是客户的信用等级,该指标作为输出层是多分类变量,可选择Softmax函数,公式见(2.9),该函数常用于多分类问题,输出层将数据映射到一个概率分布上。

学习函数定义了如何根据网络的误差来调整权重,本文采用梯度下降法来作为学习函数,梯度下降法是根据误差的梯度更新权重,是一种通用的优化算法,可以应用于多种类型的神经网络和机器学习模型,并且允许允许用户设置学习率,这是一个重要的超参数,可以控制权重更新的步长,通过调整学习率,可以平衡算法的收敛速度和稳定性。较小的学习率可能导致收敛速度较慢,而较大的学习率可能导致算法在最优解附近振荡。


针对多分类问题,通常会使用交叉熵损失函数作为误差函数,来评估模型的预测结果与真实标签之间的差异。公式见(2.8),作为模型的性能函数,本文研究的输出有二分类和多分类。不同的性能函数对应着不同的误差度量方式,如均方误差、交叉熵误差等。交叉熵误差是专门为分类问题设计的性能函数。对于多分类问题,交叉熵误差能够衡量预测概率分布与真实概率分布之间的差异,从而指导网络进行正确的分类。与均方误差等性能函数相比,交叉熵误差在训练初期不会导致梯度消失问题。这是因为交叉熵误差的梯度与激活函数的导数无关,从而避免了因激活函数饱和而导致的梯度消失。而且交叉熵误差的梯度计算简单且稳定,网络在训练过程中能够更快地收敛到最优解。


若没有达到预设的迭代次数或误差没有达到预设的最小值,需更新网络的权值和阈值,输入层与隐藏层权值$\omega_{ij} $更新公式见(3.8),隐藏层与输出层权值$\omega_{jk} $更新公式见(3.9),其中$\eta$为学习率。

\begin{equation}
	\omega_{ij} = \omega_{ij} + \eta H_j(1 - H_j)x(i) \sum\limits_{k=1}^{m}\omega_{jk}e_k  \quad i =1,2,...,n;j =1,2,...,l
\end{equation}

\begin{equation}
	\omega_{jk} = \omega_{jk} + \eta H_j(1 - H_j)e_k  \quad j =1,2,...,l; k =1,2,...,m
\end{equation}

输入层与隐藏层的阈值$a_j$更新公式见(3.10),隐藏层与输出层阈值$b_j$更新公式见(3.11),其中$\eta$为学习率,$e_k$为误差。
\begin{equation}
	a_j = a_j + \eta H_j(1 - H_j)\sum\limits_{k=1}^{m}\omega_{jk}e_k  \quad j =1,2,...,l
\end{equation}

\begin{equation}
	b_k = b_k + e_k  \quad k =1,2,...,m
\end{equation}


 \section{本章小结}
 本章是模型构建章节,主要是说明网络的构建模型,具体的训练过程将放置第四章节中进行。
本章主要内容是基于广泛的文献研究、权威机构的报告以及A公司现有的信用评价系统,进行了评价指标的初步筛选。利用二元Logistic回归算法及SPSS工具,通过单因素及多因素相结合的方式,筛选出在财务、业务、企业资质方面的6个影响客户未来是否会逾期的关键指标,在单因素分析中,逐一考察了每个指标与客户是否逾期之间的关联性,通过统计检验确定每个指标对客户逾期行为的预测能力。在多因素分析中,综合考虑多个指标之间的相互作用,通过构建多元回归模型,识别出对客户逾期行为具有显著影响的指标组合。

基于之前分析筛选出的关键评价指标,进一步利用BP神经网络构建了信用评价模型,期望预测出客户未来的逾期情况及客户的信用等级。在模型构建过程中,设计了网络结构,确定输入层、隐藏层和输出层的神经元数量及连接方式。为了使网络能够快速、准确地学习并优化,使用梯度下降法等优化算法对网络进行训练,这些算法能够通过不断迭代,调整网络中的权重和阈值,使模型的预测结果与实际数据之间的误差逐渐减小。