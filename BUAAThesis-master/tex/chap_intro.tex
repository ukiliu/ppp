% !TeX root = ../Template.tex
% [绪论]
\chapter{绪论}

%%============================
\section{选题背景和意义}
\subsection{选题的背景}
随着全球性自由市场经济的迅速发展,企业的销售方式发生巨大的变化,由过去单一的计划客户转变为多元的信用销售为主。当企业产成品存货较多时,一般都采用较为优惠的信用条件进行销售,把存货转化为应收账款,减少产成品存货,节约相关的开支。然而,信用销售起到促进销售的同时,也会给企业在偿债能力以及应收账款的管理和现金流方面带来很大的挑战,稍有管理不善就容易因客户不能还款而导致坏账发生。因此,加强对客户的客户信用管理显得越来越重要。

信用评级是用于衡量企业信用风险的工具,是作为企业商业活动或是在金融市场上的重要参考指标。通过评估企业的等级,可以了解该企业的信用质量和偿还能力。客户信用评级是企业为了有效控制客户信用风险,实现信贷资金的安全性、流动性和收益性。

A公司目前上千家客户,覆盖全球众多国家及地区,客户信用质量关系到企业经营成败,所以需要密切关注客户的信用状况,并采取相应的客户信用等级评价措施来实时掌握客户的信用状况,制定相应的还款账期,来降低逾期还款风险、保护企业的利益。

\subsection{选题的意义}
客户信用评级下降,意味着客户的偿付能力可能会降低,企业可能面临逾期付款或坏账的风险,从而对企业的财务状况产生不利影响。通过合理的信用风险评级,不仅可以得到客户信用等级的排序,为公司进行业务来往提供决策参考;还能确定每个信用等级贷款客户的违约损失率大小,为公司对每个客户的贷款额度的设定提供依据。对与公司来说,客户信用评价的意义,具体可以表现为以下几个方面:

1、风险控制。客户信用评价可以帮助企业识别潜在的信用风险客户,包括那些可能逾期付款或违约的客户,从而帮助企业提前采取风险控制措施,避免与高风险客户进行交易,从而降低逾期付款和坏账损失的发生。同时能帮助企业制定相应的风险管理策略,包括建立合理的信用额度、采取灵活的付款方式等,避免过度放宽信用条件,从而提高资金利用效率。

2.、收益提升。客户信用评价可以帮助企业识别高质量的客户,从而优化客户选择,帮助企业建立更加稳定的合作关系;通过信用评价也可以帮助企业了解客户的信用状况和偏好,从而优化销售策略,制定更加精准的营销计划,提高销售收入和市场份额。

3、决策支持。客户信用评价可以帮助企业根据客户的信用状况和历史付款记录,制定合理的信用额度和付款条件。

4、 管理优化。企业可以建立完善的客户信用评价体系,引入自动化的客户信用评价流程,提高评价效率,减少人为因素的干扰,确保评价结果的客观性和一致性。

通过以上方式,企业可以实现对客户信用评价的管理优化,提高评价的准确性和效率,降低信用风险,同时更好地支持企业的决策和运营,提高客户满意度和忠诚度,从而增强企业的竞争力。

%%============================
\section{国内外研究现状}

\nocite{*}
在信用评价指标体系选择的研究中,针对企业或客户的信用评价通常基于财务因素和非财务因素的综合考量,这些指标的选择会根据合作企业的特定特点进行调整。基于这些评估结果,进一步通过信用得分或违约概率的量化手段,将企业或客户的信用状况划分为不同的信用等级。ChenY.S.et al.(2024)基于各等级违约概率(PD)的差异.将企业划分为5 个信用等级。Roy P.K.(2023)研究提出了一个易于实施的ESG信用评级模型,它考虑了ESG和财务表现来填补现有的研究空白,提高了整体经济能力,同时最小化环境、社会和治理风险。Duan JC et al.(2021)基于上市公司违约概率(PD)将上市公司划分为 AAA、AA+等 21 个等级。Serano-Cinca C.et al.(2016) 基于企业的 FICO分数将企业划分为7个信用等级。霍思彤等(2022)从财务指标和非财务指标两个维度设计7个一级指标和25个二级指标,采用专家打分和层次分析法确定每个指标的权重,并借鉴已有研究选取了基于企业的偿债能力、经营能力、盈利能力、发展能力四个方面共13项指标构成中小企业信用评价指标体系。

徐德顺等(2021)基于复杂系统理论和三重积分原理,从主观履约意愿、客观履约能力及外部履约环境三个维度设计指标体系,构建WCE三维信用评价模型,将企业信用评价拓展至三维空间内。李晓寒(2021)利用公司内部财务信息建立 CR-CS 模型(信用评级—资本结构模型),对中国信用评级是否会影响非金融上市公司的资本结构展开研究。
罗勇和陈治亚(2015)基于模糊综合法构建了供应链金融客户信用评价指标体系。通过对供应链金融进行信用风险分析,确定了信用评价指标体系和评价标准。采用实例验证了评价指标体系的计算结果。该指标体系由客户企业信用、供应链信用和客户企业所在地信用状况等3个一级指标构成,其中客户信用状况包括4个二级指标,供应链信用状况包括等4个二级指标,地区信用状况包括2个二级指标,共计10个二级指标;采用专家咨询法结合层次分析法确定了指标权重和评价标准。Mileris R(2012)通过因子分析与 Probit 模型进行指标筛选,确定了包括工业生产指数等影响信用风险的关键指标。Dainelli,et al.(2013)通过 logistic 模型建立了包括盈利能力、偿付能力和信贷质量等指标的中小企业信用评级体系。Siyuan Chen(2023)以logistic模型和多元线性规划为基础,分析了影响企业信用风险的因素,通过logistic模型预测了各企业的违约风险指数。 Abid et al.(2022) 采用logistic回归模型来评估服务业企业支付违约风险的决定因素,赵志冲(2018)在筛选与违约状态显著相关的指标时,使用二元Logistic逻辑回归算法建立筛选模型,准确率达到95.6\%。
 
 
在信用评价模型构建方面,模型是多样化的,不同的模型针对特定的评估需求和应用场景进行设计,采用独特的评估方法和技术。有基于统计学的信用评价模型、基于机器学习的信用评价模型,还有一些针对特定行业的信用评价模型。姚定俊等(2023)从中小微企业的基本情况出发,建立有针对性的信用风险评价指标体系; 选用 SVM 作为违约风险估计的基础模型,以RF算法筛选指标体系来提升 SVM 模型的分类能力。在参数寻优上,初步考虑使用 SMA 算法,并对SMA进行改进,最终形成RF-LSMA-SVM模型进行中小微企业信用风险评价,进一步提升了信用风险判定的准确性和实用性。

阳彩霞(2022)采用BP神经网络的数据挖掘算法建立一个供应商信用评价模型,在测试数据集基础上采用混淆矩阵对模型进行检验,该模型的准确度高达百分之96.7。刘英杰和李光等(2022)针对当前工程监理市场存在的企业失信、恶意竞争、违规投标等信用缺失现象,结合工程监理行业的发展,构建了一套有效合理的工程监理企业信用评价指标体系,并结合实例分析,采用 AHP 和改进灰色关联模型的方法对其信用指标进行赋权,确定了工程监理企业的信用评价等级。曹轲(2022)IGWOBPDA算法通过结合改进的灰狼算法和BP神经网络,针对无线传感器网络数据融合中的关键问题提出了有效的解决方案,并通过仿真实验验证了其性能优势。任向英(2022)针对中小企业经营中的信用风险,提出灰色聚类评估方法,重点分析该评估法的原理与应用方式,并结合实证分析,评估中小企业的信用情况。

杨莲和石宝峰等(2022)利用 ClassBalancedLoss中的平衡因子ω弥补交叉熵函数无法调节两类样本损失权重的缺陷,克服由样本不均衡带来的评价模型对非违约样本识别过度、对违约样本识别不足问题。并通过考虑数据重叠,利用随机覆盖方法进行不放回采样,将图像识别领域中的ClassBalancedLoss函数引入信用评价领域。董秉坤等(2022)研究基于AHP-熵值组合赋权法的零售户信用评价模型。该模型采用主观赋权法(AHP)和客观赋权法(熵权法)相结合的组合赋权方法,以弥补单一赋权带来的不足,力求将主观随机性控制在一定范围内,确保主客观赋权中的中正,实现了零售户信用的定性和定量评估的有效结合。王灿(2022)运用 EasyEnsemble 方法解决信用数据集不平衡问题,再通过非对称误差成本的核 SVM、逻辑斯蒂回归、带有距离加权的KNN 算法以及C5.0算法的决策树的 Bagging 集成得到个人组合评价模型。郭鑫等(2022)提出层次分析法的科研信用评价模型。采用空间分布式信息重构方法进行科研信用评价的模糊特征重组,进行科研信用多元评价的信息挖掘和特征提取与进行科研信用评价的均衡配置和线性规划设计,实现科研信用评价的层次化分析和特征重建。刘翠玲和胡聪等(2022)构建了一种基于XGB算法的客户多维信用评价模型,并基于特征重要度方法进行特征选择,采用极限梯度提升方法以及树模型对客户信用进行模型构建,计算不同节点上不同的增益值来获取最佳的预测效果,从而构建一个准确、稳定的客户信用评价模型。郑晓杰和王双(2019)采用德尔菲法、层次分析法、指数分析法、模糊综合分析相结合的方式,针对M银行对中小型企业客户的信用评价进行模型构建,并建立起一套科学、可行的中小企业信用评价体系。温小霓和韩鑫蕊(2017)针对科技型中小企业,首先建立了一个多层次的信用风险评价指标体系,并利用t检验和因子分析对指标体系进行简化和降维,然后在此基础上构建基于MLP神经网络的科技型中小企业信用风险评价模型。

周帆等(2022)指出卷积神经网络(CNN)通过捕捉金融市场的复杂模式和趋势,能够有效辅助投资者和金融机构做出更加明智的决策,在金融领域的时间序列预测中发挥着重要作用,它被广泛应用于多种金融场景,包括金融风险的综合评估、市场波动性的评估、交易欺诈等行为的检测。钟世钦(2022)介绍了一种融合了卷积神经网络(CNN)与通用机器学习算法,实验结果显示,该模型在网贷风险预测方面表现出显著的优越性,同时证实了CNN能够有效提升一般机器学习模型的性能。这一发现为网贷风险预测领域提供了一种全新的思路和解决方案。王鑫和王莹(2022)提出了一基于LSTM-CNN模型的中小企业信用风险预测方法,使用该模型自动从数据中提取与信用风险相关的特征,并据此进行准确的信用风险预测。发现基于LSTM-CNN的预测模型在性能上显著优于其他信用风险预测模型。
Baser等(2023)提出一种基于聚类的模糊分类(CBFC)的信用风险评估方法,将模糊聚类与监督学习相结合的策略来构建信用评分模型。Roy Pranith Kumar(2022)早期的可持续信用评分模型被扩展为一个基于ESG的信用评级模型,以对借款人进行分类。Hamido Fujita等(2022)将企业信用状况分为低违约风险、中等违约风险和高违约风险三类。在采用OVO分解融合方法处理多类问题的前提下,提出了两种新的多类不平衡企业信用评价模型,即OVO-AB-OVO-LightGBTGBM-M1模型和OVO-AB-GBM-M2模型,将AB不平衡处理方法与LightGBM集成分类器相结合。
Yujia Chen, Raffaella Calabrese(2023)等人通过分析局部可解释模型无关解释和局部解释解释(LIME),并研究了它们在不同的类不平衡水平上的解释性能。使用由欧洲数据中心提供的住宅抵押贷款数据和另外两个开源的信用评分数据集来验证结果的稳健性。XGBoost和随机森林被选为“黑盒”机器学习模型来生成预测。
Furkan Baser(2022)等提出一种基于聚类的模糊分类(CBFC)的信用风险评估方法,该方法根据FKM聚类分布给聚类,通过提高ML算法的预测能力。通过对具有模糊隶属度值的模型输出进行加权,开发一种计算每个输入的单一违约概率(PD)的算法。构建了模糊聚类与监督学习相结合策略的信用评分模型。
Boughaci等人(2021)在估计随机森林集成模型之前也使用k-means聚类来提高财务困境预测性能。Golbayani等(2020)和Terry Harris(2013) 使用了通过支持向量机的方法筛选出信用风险评价指标,Golbayani等并对机器学习技术预测信用评级的文献结果进行了调查和比较分析,袋装决策树、随机森林、支持向量机和多层感知机,使用10倍交叉验证技术评估结果,得出基于决策树的模型具有优越的性能。

Bao人(2019)提出了提出了一种将无监督学习与监督学习相结合的信用风险评估策略,提出一种基于聚类和分类方法相结合的算法来评估个体的信用风险。在此背景下,作者讨论了7种监督分类模型以及两种不同的聚类模型,k-均值和自组织映射。

Qian, Hongyi(2023)等提出一种新颖的端到端软重排序一维卷积神经网络(SR-1D-CNN),该模型的核心在于其软重排序机制,该机制能够有效提升CNN对于不同规模数据集的分类性能。实验结果表明,随着数据规模的增加,CNN相较于其他基准模型展现出了更优越的分类效果。Xu, Meiling等(2022)提出了CNN和LightGBM的组合算法,建立了带有App行为的信用评分模型,对网络信用贷款违约风险进行评估。研究表明综合模型与传统评分模型的对比表明,综合模型在分类性能上取得了显著提升,App行为可以作为传统信用评分模型的有力补充。Osei, Salomey等人(2021)通过关注信用风险的多层感知器(MLP)和卷积神经网络(CNN)来研究信用风险的机器学习模型。通过对模拟数据进行了压力测试,发现与CNN模型相比,MLP模型的表现并不好,准确率为43\%,而训练期间的准确率为89\%。得出CNN模型能够在压力情况下表现更好的准确性和其他指标。Vardhani等人(2019)为检测和揭露信用卡欺诈提出了一种新的数据挖掘算法——CNN算法来检测信用卡欺诈。CNN算法是一种用于分类的非参数方法。CNN算法利用数据约简的概念,通过保留对决策有重要意义的样本,形成了一个浓缩集。

\section{研究问题和研究内容}
\subsection{研究问题}
A公司目前的信用评价系统是基于客户的财务报表、业务相关数据,使用业务制定的评分规则来计算出相应的信用评分,在法务层面是人工审核企业信息并给予主观评分,系统得出评分后,后续会有主观的授信建议阶段,最终的评价模型涉及到的指标较多,而且存在主观因素。

随着业务的发展,该模型所得出的评价信息与实际客户的应收账款逾期情况逐渐趋于不一致情况,信用评级较高的客户应收账款逾期的情况发生概率并不低。同时经查看数据客户发生逾期的情况在各评级等级中发生概率差别较小,无法从评级中看出客户未来是否会发生逾期。

同时,信用等级高低与未来是否会逾期的关系并不是高度相关的关系。虽然信用等级可以提供一定的预测能力,但仍需要综合考虑其他因素,以更准确地评估实体的未来信用行为。信用等级并不是预测未来的唯一依据。尽管高信用等级的企业客户未来逾期的风险较低,但仍有可能受到市场变化、突发事件、政策调整等因素的影响而出现逾期行为。同样,低信用等级的实体也有可能通过改善经营管理、加强信用管理等方式提高自身的信用状况,降低未来逾期的风险。

所以本次研究的问题是期望将影响企业客户实际违约的因素挖掘出来,计算出一套更为客观的信用评价体系,在能看到客户信用评价的同时,也能预测到该客户未来的逾期情况。同时使评级模型脱离主观评价,使客户的评级更为客观。
\subsection{研究内容}
本论文的主要研究内容如下:

(1)针对当前A公司在企业客户信用等级评价上遇到的问题,结合公司的业务背景及国内外相关文献的研究,进行影响企业客户信用指标的分析,预计采用二元Logistic逻辑回归,并在财务相关指标、业务相关指标、及企业资质等维度的相关指标分析影响客户未来逾期的情况。

(2)基于构建出的信用评价指标体系,结合国内外相关的理论研究,拟采用BP神经网络及卷积神经网络(CNN)来分别进行信用评价模型的建立,通过比对两个模型的验证结果,企业客户的信用等级和未来的逾期情况。
在构建完成的信用评价指标体系基础上,为了更全面、精准地进行信用评价,我们计划结合国内外先进的理论研究成果,分别采用BP神经网络和卷积神经网络(Convolutional Neural Network, CNN)来分别构建和优化信用评价模型。通过这两种不同的神经网络架构,探索不同数据特征对信用评价的影响,并期望通过模型的比较,找到最适合当前数据环境和业务需求的信用评价解决方案。

(3)开展模型验证及结果分析。以A公司近期的业务数据作为验证数据,结合构建的信用等级评价模型进行应用实现,分析构建的两个模型分别在A公司实际业务中的表现,对预测结果进行深入的解读和比较。基于这些分析结果,得出两个关于模型在A公司信用评价方面适用性和准确性的结论。

\subsection{技术路线}
本文通过理论研究与实证研究相结合方法 , 研究了企业客户的信用风险评级指标体系的构建 , 评价模型的建立, 研究的技术路线如下图1所示。

\begin{figure}[!h]
	\centering
	\includegraphics[width=.7\linewidth]{../../../../../pictures/jishuluxian.png}
	\caption{技术路线图}
\end{figure}

\section{创新点}
本文的创新点有以下两点:

(1)在评价模型中增加企业客户法务相关的指标,法务因素对企业的信用评价有着重要的影响。目前A公司中的法务指标均为法务部门相关人员手动打分,存在较大主观性,此研究中将引入法务指标,研究这些法务指标对客户信用的影响性。

(2)此研究的信用评价模型得出的是客户的信用等级和未来是否会逾期这两个结果。在查看客户的等级的同时也能观测到该客户未来的信用情况,从而可以给该客户制定相应的应收账款的账期,减少逾期的发生。

(3)使用真实的客户逾期数据,逾期数据是信用评价模型中的重要指标之一,能够直接反映客户的信用风险和还款能力。使用真实的客户逾期数据可以帮助构建信用评价模型的训练和验证数据集。通过对逾期数据进行特征提取和模型训练,可以建立更贴近实际情况的信用评价模型,提高模型的预测准确性和泛化能力。同时通过使用逾期数据作为测试集,可以对模型的准确率指标进行评估,从而验证模型的预测能力和稳健性。

\section{章节结构}
本文分为四个章节,各章节研究框架与内容安排如下:

第一章节为绪论部分。本部分依次对本研究的研究背景和意义、国内外文献研究、研究技术路线、研究内容、创新点进行阐述。

第二章为本文模型构建的理论部分,分别对本文中用到的二元Logistic回归、BP神经网络、卷积神经网络在模型构建过程中用到的核心理论及技术点进行阐述。

第三章聚焦于信用评价模型的构建过程。首先通过综合文献、权威机构推荐及A公司现有系统,筛选出关键评价指标。然后利用二元Logistic回归模型对这些指标进行筛选优化。最后分别定义了BP神经网络和卷积神经网络(CNN)的模型架构及关键变量,并进行了模型架构的详细设计和参数的初始化。

第四章是基于构建的评价模型,使用A公司企业客户的数据进行实证分析,先通过二元Logistic回归筛选出影响企业客户是否逾期的评价指标。再分别使用BP神经网络和CNN模型进行训练,预测出客户的信用等级和未来是否逾期,两模型得出的检验结果与实际情况进行对比分析,得出更适合A公司信用评价业务的模型。

最后为本文的结论部分,在该章节中将对研究成果进行总结,并深入剖析研究中存在的不足,同时提出相应的改进建议,以期为未来的研究和实践提供有价值的参考。