% !TeX root = ../Template.tex
% [绪论]
\chapter{绪论}

%%============================
\section{选题背景和意义}
\subsection{选题的背景}
随着全球性自由市场经济的迅速发展,企业的销售方式发生巨大的变化,由过去单一的计划客户转变为多元的信用销售为主。当企业产成品存货较多时,一般都采用较为优惠的信用条件进行销售,把存货转化为应收账款,减少产成品存货,节约相关的开支。然而,信用销售起到促进销售的同时,也会给企业在偿债能力以及应收账款的管理和现金流方面带来很大的挑战,稍有管理不善就容易因客户不能还款而导致坏账发生。因此,加强对客户的客户信用管理显得越来越重要。

信用评级是用于衡量企业信用风险的工具,是作为企业商业活动或是在金融市场上的重要参考指标。通过评估企业的等级,可以了解该企业的信用质量和偿还能力。客户信用评级是企业为了有效控制客户信用风险,实现信贷资金的安全性、流动性和收益性。

A公司目前上千家客户,覆盖全球众多国家及地区,客户信用质量关系到企业经营成败,所以需要密切关注客户的信用状况,并采取相应的客户信用等级评价措施来实时掌握客户的信用状况,制定相应的还款账期,来降低逾期还款风险、保护企业的利益。

\subsection{选题的意义}
客户信用评级下降,意味着客户的偿付能力可能会降低,企业可能面临逾期付款或坏账的风险,从而对企业的财务状况产生不利影响。通过合理的信用风险评级,不仅可以得到客户信用等级的排序,为公司进行业务来往提供决策参考;还能确定每个信用等级贷款客户的违约损失率大小,为公司对每个客户的贷款额度的设定提供依据。对与公司来说,客户信用评价的意义,具体可以表现为以下几个方面:

1、风险控制。客户信用评价可以帮助企业识别潜在的信用风险客户,包括那些可能逾期付款或违约的客户,从而帮助企业提前采取风险控制措施,避免与高风险客户进行交易,从而降低逾期付款和坏账损失的发生。同时能帮助企业制定相应的风险管理策略,包括建立合理的信用额度、采取灵活的付款方式等,避免过度放宽信用条件,从而提高资金利用效率。

2.、收益提升。客户信用评价可以帮助企业识别高质量的客户,从而优化客户选择,帮助企业建立更加稳定的合作关系;通过信用评价也可以帮助企业了解客户的信用状况和偏好,从而优化销售策略,制定更加精准的营销计划,提高销售收入和市场份额。

3、决策支持。客户信用评价可以帮助企业根据客户的信用状况和历史付款记录,制定合理的信用额度和付款条件。

4、 管理优化:企业可以建立完善的客户信用评价体系,引入自动化的客户信用评价流程,提高评价效率,减少人为因素的干扰,确保评价结果的客观性和一致性。

通过以上方式,企业可以实现对客户信用评价的管理优化,提高评价的准确性和效率,降低信用风险,同时更好地支持企业的决策和运营,提高客户满意度和忠诚度,从而增强企业的竞争力。

%%============================
\section{国内外研究现状}

\subsection{国内研究现状}
\nocite{*}
在信用评价指标体系选择研究方面,对企业或者客户信用评价的研究上,一般是基于财务或者非财务的指。\citet{YZJI202206016}根据合作企业特点,从财务指标和非财务指标两个维度设计7个一级指标和25个二级指标,采用专家打分和层次分析法确定每个指标的权重,建立起6级信用等级体系。\citet{XDBY202105045}基数据的可获得性、低相关性以及全面性等考虑,借鉴已有研究选取了基于企业的偿债能力、经营能力、盈利能力、发展能力四个方面共13项指标构成中小企业信用评价指标体系。徐德顺和马凡慧等(2021)基于复杂系统理论和三重积分原理,从主观履约意愿、客观履约能力及外部履约环境三个维度设计指标体系,构建WCE三维信用评价模型,将企业信用评价拓展至三维空间内。李晓寒(2021)基于 Kisgen 的研究方法,利用公司内部财务信息建立 CR-CS 模型(信用评级—资本结构模型),对中国信用评级是否会影响非金融上市公司的资本结构展开研究。
 

罗勇和陈治亚(2015)基于模糊综合法构建了供应链金融客户信用评价指标体系。通过对供应链金融进行信用风险分析,确定了信用评价指标体系和评价标准。采用实例验证了评价指标体系的计算结果。该指标体系由客户企业信用、供应链信用和客户企业所在地信用状况等3个一级指标构成,其中客户信用状况包括4个二级指标,供应链信用状况包括等4个二级指标,地区信用状况包括2个二级指标,共计10个二级指标;采用专家咨询法结合层次分析法确定了指标权重和评价标准。

在信用评价模型构建方面,模型是多样化的,不同的模型针对特定的评估需求和应用场景进行设计,采用独特的评估方法和技术。有基于统计学的信用评价模型、基于机器学习的信用评价模型,还有一些针对特定行业的信用评价模型。周颖和张志鹏(2023)提出了新的信用等级划分约束条件,确保违约企业更多分布在较低等级,改变了现有研究将违约企业更多划分在较高等级的不合理现象。通过巴塞尔协议关于预期损失与违约风险暴露、违约概率、违约损失率的关系,近似得出上市公司的预期损失,给出了解决上市公司缺乏违约损失数据的方法。姚定俊和顾越等(2023)从中小微企业的基本情况出发,建立有针对性的信用风险评价指标体系; 从财务风险的视角选取财务指标,并创新性地将中小微企业管理层特征纳入信用风险指标体系。选用 SVM 作为违约风险估计的基础模型,以RF算法筛选指标体系来提升 SVM 模型的分类能力。在参数寻优上,初步考虑使用 SMA 算法,并对SMA进行改进,最终形成RF-LSMA-SVM模型进行中小微企业信用风险评价,进一步提升了信用风险判定的准确性和实用性。华泰证券固定收益部课题组(2023)将大数据、人工智能等技术引入信用评价与投资管理领域,通过多源数据融合方案和数据自动化校验,提高数据质量。在获取数据的基础上,构建智能信用评价模型动态评估发债主体和交易对手的信用情况,提升金融机构的风险管控能力。信用评价模型开发需经过建模准备、单因素分析、多因素分析、回归模型建立和验证五个步骤。

阳彩霞(2022)采用BP神经网络的数据挖掘算法建立一个供应商信用评价模型,在测试数据集基础上采用混淆矩阵对模型进行检验,该模型的准确度高达百分之96.7。刘英杰和李光等(2022)针对当前工程监理市场存在的企业失信、恶意竞争、违规投标等信用缺失现象,结合工程监理行业的发展,构建了一套有效合理的工程监理企业信用评价指标体系,并结合实例分析,采用 AHP 和改进灰色关联模型的方法对其信用指标进行赋权,进而确定工程监理企业的信用评价等级。曹轲(2022)IGWOBPDA算法通过结合改进的灰狼算法和BP神经网络,针对无线传感器网络数据融合中的关键问题提出了有效的解决方案,并通过仿真实验验证了其性能优势。任向英(2022)针对中小企业经营中的信用风险,提出灰色聚类评估方法,重点分析该评估法的原理与应用方式,并结合实证分析,评估中小企业的信用情况。

杨莲和石宝峰等(2022)利用 ClassBalancedLoss中的平衡因子ω弥补交叉熵函数无法调节两类样本损失权重的缺陷,克服由样本不均衡带来的评价模型对非违约样本识别过度、对违约样本识别不足。并通过考虑数据重叠,利用随机覆盖方法进行不放回采样,将图像识别领域中的ClassBalancedLoss函数引入信用评价领域。董秉坤和郑陈柔雨等(2022)研究基于AHP-熵值组合赋权法的零售户信用评价模型。该模型采用主观赋权法(AHP)和客观赋权法(熵权法)相结合的组合赋权方法,以弥补单一赋权带来的不足,力求将主观随机性控制在一定范围内,确保主客观赋权中的中正,实现了零售户信用的定性和定量评估的有效结合。王灿(2022)运用 EasyEnsemble 方法解决信用数据集不平衡问题,再通过非对称误差成本的核 SVM、逻辑斯蒂回归、带有距离加权的KNN 算法以及 C5.0算法的决策树的 Bagging 集成得到个人组合评价模型。郭鑫(2022),提出层次分析法的科研信用评价模型。采用空间分布式信息重构方法进行科研信用评价的模糊特征重组,进行科研信用多元评价的信息挖掘和特征提取与进行科研信用评价的均衡配置和线性规划设计,实现科研信用评价的层次化分析和特征重建。采用二乘规划模型进行科研信用评价的大数据调度和模糊空间信息采样,结合资源优化调度方法,实现科研信用资源信息优化配置,通过博弈均衡控制方法,进行科研信用的优化评价和参数自适应评估。刘翠玲和胡聪等(2022)构建了一种基于XGB算法的客户多维信用评价模型,并基于特征重要度方法进行特征选择,采用极限梯度提升方法以及树模型对客户信用进行模型构建,计算不同节点上不同的增益值来获取最佳的预测效果,从而构建一个准确、稳定的客户信用评价模型。

程砚秋和徐占东(2019),借鉴ELECTREIII的评价原理,计算新增贷款客户的信用风险评价得分。以违约、非违约样本加权后的组内差异程度为基础,确定ELECTREIII的否决阈值;反映了不同评价指标数据差异大小对评价结果的影响,避免了现有阈值人为主观确定的不足。郑晓杰和王双(2019)采用德尔菲法、层次分析法、指数分析法、模糊综合分析相结合的方式,针对M银行对中小型企业客户的信用评价进行模型构建,并建立起一套科学、可行的中小企业信用评价体系。
赵志冲和迟国泰(2017)通过对比只含有某一个指标的完整模型的对数似然值和只含常数项的零模型的对数似然值确定统计量,构造了某一特定指标与违约状态之间的逻辑回归方程。比较在有无某指标时的两个对数似然值之间的差。最终建立了包括超速动比率、近三年企业授信情况、城市居民人均可支配收入等 16 个指标的小企业信用风险评价指标。温小霓和韩鑫蕊(2017)针对科技型中小企业,首先建立了一个多层次的信用风险评价指标体系,并利用t检验和因子分析对指标体系进行简化和降维,然后在此基础上构建基于MLP神经网络的科技型中小企业信用风险评价模型。

\subsection{国外研究现状}
在信用评价指标体系选择研究方面,基于信用得分或违约概率的信用等级划分。Roy P.K.(2023)研究提出了一个易于实施的ESG信用评级模型,它考虑了ESG和财务表现来填补现有的研究空白,提高了整体经济能力,同时最小化环境、社会和治理风险。Duan JC 和 Li 基于上市公司违约概率(PD)将上市公司划分为 AAA、AA+等 21 个等级。
%https://readpaper.com/paper/3120943764 用该地址生成参考文献
 Serano-Cinca C.和 Cutierrez-Nieto B 基于企业的 FICO分数将企业划分为7个信用等级。
ChenY.S和 Cheng C. H.基于各等级违约概率(PD)的差异.将企业划分为5 个信用等级。Moon T.Kim Y.等根据企业的信用得分,划为 10 个信用等级。
Mileris R(2012)通过因子分析与 Probit 模型进行指标筛选,确定了包括工业生产指数等影响信用风险的关键指标。
Dainelli,et al.(2013)通过 logit 模型建立了包括盈利能力、偿付能力和信贷质量等指标的中小企业信用评级体系。

在信用评价模型构建方面,
Baser等(2023)提出一种基于聚类的模糊分类(CBFC)的信用风险评估方法,将模糊聚类与监督学习相结合的策略来构建信用评分模型。Roy Pranith Kumar(2022)早期的可持续信用评分模型被扩展为一个基于ESG的信用评级模型,以对借款人进行分类。Hamido Fujita等(2022)将企业信用状况分为低违约风险、中等违约风险和高违约风险三类。在采用OVO分解融合方法处理多类问题的前提下,提出了两种新的多类不平衡企业信用评价模型,即OVO-AB-OVO-LightGBTGBM-M1模型和OVO-AB-GBM-M2模型,将AB不平衡处理方法与LightGBM集成分类器相结合。
Yujia Chen, Raffaella Calabrese(2023)等人通过分析局部可解释模型无关解释和局部解释解释(LIME),并研究了它们在不同的类不平衡水平上的解释性能。使用由欧洲数据中心提供的住宅抵押贷款数据和另外两个开源的信用评分数据集来验证结果的稳健性。XGBoost和随机森林被选为“黑盒”机器学习模型来生成预测。
Furkan Baser(2022)等提出一种基于聚类的模糊分类(CBFC)的信用风险评估方法,该方法根据FKM聚类分布给聚类,通过提高ML算法的预测能力。通过对具有模糊隶属度值的模型输出进行加权,开发一种计算每个输入的单一违约概率(PD)的算法。构建了模糊聚类与监督学习相结合策略的信用评分模型。
Boughaci等人(2021)在估计随机森林集成模型之前也使用k-means聚类来提高财务困境预测性能。Golbayani等(2020)对应用机器学习技术预测信用评级的文献结果进行了调查和比较分析,将之前研究中被认为有用的四种机器学习技术(袋装决策树、随机森林、支持向量机和多层感知机)应用于相同的数据集。我们使用10倍交叉验证技术评估结果,得出基于决策树的模型具有优越的性能。

Bao人(2019)提出了提出了一种将无监督学习与监督学习相结合的信用风险评估策略,提出一种基于聚类和分类方法相结合的算法来评估个体的信用风险。在此背景下,作者讨论了7种监督分类模型以及两种不同的聚类模型,k-均值和自组织映射。
Terry Harris(2013)通过支持向量机的方法遵选出信用风险评价指标,构建了巴巴多斯信贷联盟的信用评价模型。


\subsection{文献综述总结}
在信用评价体系的筛选上,应用较为广泛的方法是偏相关分析、支持向量、因子分析与 Probit 模型等方法,这些方法受各个因素的影响,比如各个企业的特殊性,行业标准差异性,或者受限于相关数据的可获得性等,这些对因素对评价体系筛选的准确性有待验证,在筛选指标时有些方法像模糊层次分析法存在主观性,导致结论存在客观偏差。这些体系在实际应用中存在一些不足,企业信用评级模型需要考虑到不同行业间的经营特色,企业经营状况等因素,可以考虑加入企业的业务往来数据作为参考指标,提升评价结果的准确性。



在信用评价模型方面,选择预测信用评价的算法时,需要根据实际情况综合考虑各种因素。对于简单的二分类问题,逻辑回归可能是一个不错的选择;对于需要解释性的场景,决策树可能更合适;对于需要高预测准确率的场景,随机森林和支持向量机可能是更好的选择;而对于数据量大、关系复杂的场景,神经网络可能具有更好的性能。MLP神经网络,持向量机,Logistic 回归信用评价模型,
AHP 加熵权法,ESG,SVM 等方法在研究上应用较广泛,其中机器学习的相关方法能
更为客观地度量企业真实的信用质量,基于机器学习的信用评价模型则更加侧重于对数据的挖掘和深度分析。通过运用复杂的算法和模型,机器学习模型能够自动从数据中提取有用的信息,并对信用风险进行更为精准和个性化的评估。这类模型在处理非线性关系、高维度数据以及动态变化方面具有优势,适用于复杂多变的信用评估场景。且基于BP神经网络的数据挖掘算法,模型的准
确度高。随机森林在信用评估中表现出较高的分类准确率和稳定性,并且具有较高的噪声容忍度,能够有效避免过拟合现象。

在信用等级划分方面,在不同的研究中,企业信用等级的划分标准和方法存在差
异,不具备统一的标准和一致性。不同研究结果都是基于不同的行业或者业务模式下
构建的划分方法,具有不可比性。如果是专业的测评机构需要选择具有统一标准的划
分方法,但是对于企业来说,可以根据与客户的业务往来状况来选择适合自己业务的
划分方式。

综上所述,在新的研究中,对于A公司来说,在指标的筛选上,可以在采用财
务相关指标的基础上,加入 A 公司企业客户的业务往来相关的业务指标来研究这些指
标对评价结果的影响。在筛选评价模型算法时,可以考虑使用机器学习算法,如 BP神经网络和随机森林来进行信用评价体系的构建,减少主观性,通过两个模型的对比筛选存储更适合A公司的评价模型。

\section{研究问题和研究内容}
\subsection{研究问题}
A公司目前的信用评价系统是基于客户的财务报表、业务相关数据,使用业务制定的评分规则来计算出相应的信用评分,在法务层面是人工审核企业信息并给予主观评分,系统得出评分后,后续会有主观的授信建议阶段,最终的评价模型涉及到的指标较多,而且存在主观因素。

随着业务的发展,该模型所得出的评价信息与实际客户的应收账款逾期情况逐渐趋于不一致情况,信用评级较高的客户应收账款逾期的情况发生概率并不低。同时经查看数据客户发生逾期的情况在各评级等级中发生概率差别较小,无法从评级中看出客户未来是否会发生逾期。

同时,信用等级高低与未来是否会逾期的关系并不是高度相关的关系。虽然信用等级可以提供一定的预测能力,但仍需要综合考虑其他因素,以更准确地评估实体的未来信用行为。信用等级并不是预测未来的唯一依据。尽管高信用等级的企业客户未来逾期的风险较低,但仍有可能受到市场变化、突发事件、政策调整等因素的影响而出现逾期行为。同样,低信用等级的实体也有可能通过改善经营管理、加强信用管理等方式提高自身的信用状况,降低未来逾期的风险。

所以本次研究的问题是期望将影响企业客户实际违约的因素挖掘出来,计算出一套更为客观的信用评价体系,在能看到客户信用评价的同时,也能预测到该客户未来的逾期情况。同时使评级模型脱离主观评价,使客户的评级更为客观。
\subsection{研究内容}
本论文的主要研究内容如下:

(1)针对当前A公司在企业客户信用等级评价上遇到的问题,结合公司的业务背景及国内外相关文献的研究,进行影响企业客户信用指标的分析,预计采用财务相关指标、业务相关指标、及企业资质等信息分析影响客户未来逾期的情况。

(2)基于构建出的信用评价指标体系,结合国内外相关的理论研究,使用收集到A公司企业客户的相关的历史数据作为训练数据,拟采用BP神经网络及二元Logistic逻辑回归方法来进行信用评价模型的建立,得出企业客户的信用等级和未来的逾期情况。

(3)开展应用研究。以A公司近期的业务数据作为验证数据,结合构建的信用等级评价模型进行应用实现,分析结果,得出结论。

\subsection{技术路线}
本文通过理论研究与实证研究相结合方法 , 研究了企业客户的信用风险评级指标体系的构建 , 评价模型的建立, 研究的技术路线如下图所示。

\begin{figure}[!h]
	\centering
	\includegraphics[width=.7\linewidth]{../../../../../pictures/jishuluxian.png}
	\caption{技术路线图}
\end{figure}

\section{创新点}
本文的创新点有以下两点:

(1)在评价模型中增加企业客户法务相关的指标,法务因素对企业的信用评价有着重要的影响。目前A公司中的法务指标均为法务部门相关人员手动打分,存在较大主观性,此研究中将引入法务指标,研究这些法务指标对客户信用的影响性。

(2)此研究的信用评价模型得出的是客户的信用等级和未来是否会逾期这两个结果。在查看客户的等级的同时也能观测到该客户未来的信用情况,从而可以给该客户制定相应的应收账款的账期,减少逾期的发生。

(3)使用真实的客户逾期数据,逾期数据是信用评价模型中的重要指标之一,能够直接反映客户的信用风险和还款能力。使用真实的客户逾期数据可以帮助构建信用评价模型的训练和验证数据集。通过对逾期数据进行特征提取和模型训练,可以建立更贴近实际情况的信用评价模型,提高模型的预测准确性和泛化能力。同时通过使用逾期数据作为测试集,可以对模型的准确率指标进行评估,从而验证模型的预测能力和稳健性。

\section{章节结构}
本文分为四个章节,各章节研究框架与内容安排如下:

第一章节为绪论部分。本部分依次对本研究的研究背景和意义、国内外文献研究、研究技术路线、研究内容、创新点进行阐述。

第二章为本文模型构建的理论部分,分别对本文中用到的二元Logistic回归及BP神经网络在模型构建过程中用到的核心理论及技术点进行阐述。

第三章是本文的核心章节,是信用评价模型构建的过程的描写。首先基于文献、权威机构及A公司现有的信用评价系统筛选出评价指标,再使用二元Logistic回归进行单因素、多因素的指标筛选,最终筛选出影响企业客户是否逾期的评价指标。再基于已筛选出的指标对是否逾期和客户信用等级使用BP神经网络构建信用评价模型。

第四章是基于构建的评价模型,使用A公司企业客户的数据进行实证分析,通过模型预测出客户的信用等级和未来是否逾期,再与实际情况进行对比分析,进行模型的检验。

最后为本文的结论部分,在该章节中将对研究成果进行总结,并深入剖析研究中存在的不足,同时提出相应的改进建议,以期为未来的研究和实践提供有价值的参考。