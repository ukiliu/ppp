\chapter{实证研究}

\section{数据筛选与描述}

本文所研究的信用评价模型,其核心在于采用了第3.1节中详细阐述的指标评价体系。这一体系经过精心构建和深入研究,旨在更全面地反映客户的信用状况。值得注意的是,这一新体系中包含的部分指标,在A公司原有的信用评价体系中并未涉及。在A公司原有的信用评价体系中,存在部分冗余指标。这些指标在评价客户的信用状况时,其贡献度和影响力相对较低,对整体信用评价结果的贡献并不显著。因此,这些冗余指标在信用评价过程中的存在意义不大,不仅增加了评价体系的复杂性,还可能干扰评价结果的准确性。
原有评价体系虽然在一定程度上能够评估客户的信用水平,但受限于其指标选择的范围,可能无法全面捕捉客户的信用特征。而新加入的指标,不仅涵盖了更多维度的信息,还能够深入挖掘客户信用状况的内在规律和趋势。这些新增指标的存在,使得信用评价模型更加完善和科学,为A公司提供了更准确的客户信用评估依据。以下为各指标在A公司原有信用评价体系中的存在情况: 


\begin{table}[h]
	\caption{基于机器学习的A公司企业客户信用评价体系}
	\label{tab:papercomponents}
	\centering
	\begin{tabular}{ccc}
		\toprule
		{\bfseries 指标} & {\bfseries 是否为A公司元评价体系指标}   \\
		\midrule
		是否上市 & 否 \\
		经营负荷情况   & 否 \\
		现金比率    &是\\
		应收账款周转率 &是   \\
		是否有逾期记录  & 是   \\
		销售净利率 & 是 \\
		\bottomrule
	\end{tabular}
\end{table}

本研究选取A公司2019年以来A公司所有进行过信用评价的数据,包含A公司所有国内与国外的所有客户数据。经过二元Logistic回归的筛选确定的六个指标,A公司的国外客户同样能够提供这六个指标的数据,本文将这部分客户的数据也纳入BP神经网络的训练中。这一举措不仅丰富了数据集,使其更具多样性和代表性,而且有望进一步提升BP神经网络模型的准确性和泛化能力。通过融入国外客户的数据,能够更全面地捕捉不同市场环境下的信用特征,从而为A公司提供更为精准和可靠的信用风险评估服务。

收集到的初始数据集为1843条,经过去除重复及缺失数据后,共计1042条有效数据,通过对这些数据的深入分析和学习,期望构建的BP神经网络模型将能够更准确地捕捉客户信用的内在规律,为A公司提供更为精准、可靠的信用风险评估和决策支持。本文的输入特征变量只有6个,而且这6个变量对客户的信用情况具有很高的解释性和预测能力,而且这些数据均为真实业务数据,使用BP神经网络能够捕捉到这些变量中的潜在规律和模式,实现对客户信用情况的准确预测,也能为模型提供有效的学习信息。
将这些数据划分为训练集、测试集和验证集。其中,训练数据有900条,用于训练网络并调整其权重和偏置,寻找到最优参数;测试数据有100条,用于在训练过程中实时评估模型的性能;验证数据有42条,用于在第四章对模型进行最终的评估。这种数据划分方式有助于全面、客观地了解模型的性能,并为其后续的优化提供有力的支持。



\section{数据预处理}

在进行BP神经网络模型构建之前,数据预处理工作显得至关重要。为了确保神经网络训练的稳定性和高效性,需要对数据进行一系列预处理操作,包括标准化、归一化等。这些步骤能够消除不同特征之间的量纲差异,使得网络能够更容易地捕捉到数据中的内在规律。

在本文中,“是否逾期”和“信用等级”作为网络的输出层,都是分类变量。对于“是否逾期”这一变量,将此变量表示为0和1,其中0代表未逾期,1代表逾期。这种简洁的表示方法有助于神经网络更好地学习和预测逾期情况。

而对于“信用等级”这一输出层变量,本文沿用了A公司现有的评级制度。A公司的信用等级分为五个层级,从高到低分别为AAA、AA、A、B、C。为了方便神经网络处理,将这些等级转换为相应的数值表示,即1、2、3、4、5。这种数值化的表示方法不仅保留了信用等级的原始信息,还使得神经网络能够更容易地处理和理解这一输出层变量。通过这样的预处理操作,可以为后续的BP神经网络训练奠定坚实的基础,提升模型的性能和准确率。
等级设置规则和变量设置如下表:

\begin{table}[h]
	\caption{信用等级划分及变量说明}
	\label{tab:papercomponents}
	\centering
	\begin{tabular}{ccc}
		\toprule
		{\bfseries 等级} & \multicolumn{1}{c} {\bfseries 变量设置} & \multicolumn{1}{c} {\bfseries 等级含义说明}  \\
		\midrule
		AAA &  5 & 偿还债务的能力极强,基本不受不利经济环境的影响,违约风险极低。 \\
		AA & 4      &  偿还债务的能力很强,受不利经济环境的影响不大,违约风险很低。 \\
		A & 3      &  偿还债务能力较强,较易受不利经济环境的影响,违约风险较低。\\
		B & 2 &  偿还债务能力一般,受不利经济环境影响较大,违约风险一般。\\
		C & 1      &  偿还债务能力较弱,受不利经济环境影响很大,有较高违约风险。 \\
		\bottomrule
	\end{tabular}
\end{table}

现金比率、应收账款周转率、销售净利率为连续变量,需做归一化处理。其余三个输入变量均为二分类变量,分类变量的输入数据是不需要进行归一化处理的。

对于连续变量,现金比率、应收账款周转率和销售净利率,为了确保这些变量在神经网络中能够被平等对待,并防止某些特征对模型训练产生过大的影响,需要对这些连续变量进行归一化处理。归一化可以将这些变量的数值范围缩放到一个共同的标准区间[0,1],从而消除量纲差异,使得网络能够更好地捕捉数据中的内在模式。

本文中的其余三个输入变量,情况则有所不同。这些分类变量通常表示某种属性或状态的存在与否,其取值往往是离散的,如0和1。由于这些变量本身就不具有连续的数值范围,因此不需要进行归一化处理。在神经网络中,这些二分类变量可以直接以独热编码(one-hot encoding)或标签编码(label encoding)的形式输入,以便模型能够正确地处理和理解这些分类特征。通过这样的处理方式,可以确保不同类型的数据在BP神经网络模型中都得到适当的处理,从而提升模型的性能和准确性。
\section{模型训练}

学习率用于控制参数更新速度的超参数。它决定了权重调整的幅度,过大可能导致参数更新过快而错过最优解,过小可能导致收敛速度过慢。通常学习率的取值范围在0.001到0.1之间,本文设置的学习率为0.01。

迭代次数的设置是网络在停止训练之前可以进行的最大迭代次数。迭代次数过多可能导致过拟合,过少可能导致训练不足。通常,初始时可以设置较大的迭代次数,然后根据训练收敛情况逐渐减小。本文经过多轮训练测试,设置的迭代次数为1000。
%可参考这句话 同时,经过测算,训练模型在200000次后,信息传播的均方误差损失已经趋于收敛。在对比训练次数依次为200000次,300000次,400000次,…1000000次,发现 500000次后均方误差收敛下降程度变化微乎其微,因而本文模型训练次数为 500000 次。

误差阈值(通常称为“停止训练条件”或“误差目标”)是一个用于决定何时停止训练过程的参数。它代表网络训练过程中期望达到的误差水平。当网络的训练误差低于这个阈值时,训练过程将停止。本研究在实验过程中对误差阈值经过多轮的调试,初始时,设置一个相对较高的误差阈值0.01。这样的设置可以为网络提供足够的训练时间,使其能够从数据中学到更多的信息。后面根据模型结果展示的趋势进行逐步减小误差阈值,最终模型设置的误差阈值为0.001.	


\section{模型检验}

BP神经网络的训练过程是通过反向传播算法来不断调整网络中的权重和偏置,使得网络的输出尽可能接近真实值,从而实现模型的优化和学习。
本研究将使用归一化处理后的462个样本数据来进行模型的训练,使用MATLAB R2023工具箱来进行训练。

\subsection{模型准确率分析}
\begin{table}[h]
	\caption{训练数据准确率}
	\label{tab:papercomponents}
	\centering
	\begin{tabular}{cccc}
		\toprule
		{\bfseries 数据类型} &{\bfseries 输出} & \multicolumn{1}{c} {\bfseries 实际数量} & \multicolumn{1}{c} {\bfseries 预测正确数量}  \\
		\midrule
		\multirow{3}{*}{训练集} &是否逾期 &  900 & 816 \\
		&信用等级 & 900      &  794 \\
		&正确率& \multicolumn{2}{c}{89.4\%}    \\
	\midrule
		\multirow{3}{*}{测试集}&是否逾期 &  100 & 90 \\
		&信用等级 & 100      &  88 \\
	   &	正确率& \multicolumn{2}{c}{85\%}    \\
	   \midrule
	   \multirow{3}{*}{验证集}&是否逾期 &  42 & 34 \\
	   &信用等级 & 42      &  33 \\
	   &	正确率& \multicolumn{2}{c}{79.7\%}    \\
		\bottomrule
	\end{tabular}
\end{table}

使用模型训练最终得出的各组数据的准确率如上表。其中型在不同数据集上的性能表现相对一致,但也有一些细微的差别。训练集准确率较高,说明模型在训练数据上学习得较好,能够识别并正确分类大部分样本。测试集准确率略低于训练集,但仍然是一个相对较高的值。这通常意味着模型具有较好的泛化能力,能够在未见过的数据上表现良好。验证集准确率略低于测试集和训练集,但仍然是一个可接受的性能水平。验证集通常用于模型选择和超参数调整,其准确率有助于确定模型在未知数据上的预期性能。综合来看,模型的性能表现相对较好。

\subsection{模型性能分析}

\begin{figure}
	\centering
	\includegraphics[width=.7\linewidth]{../../../../../pictures/xingneng.png}
	\caption{迭代轮次性能图}
\end{figure}

模型在训练过程中会逐渐调整其参数,以最小化损失函数。该在第31轮次后达到最佳性能,如图7,说明模型在此之前一直在进行有效的学习,并且没有出现明显的过拟合或欠拟合现象。这也暗示了训练策略的有效性,包括学习率的选择、优化算法的使用以及正则化技术的应用等,都对模型的最终性能起到了积极作用。图中展示的损失误差是一个相对较低的数值,模型的输出为分类输出,此数值是可以接受的范围。模型在预测时的误差较小。这通常意味着模型能够准确地捕捉数据的内在规律和模式,从而做出准确的预测。损失误差的绝对值并不是唯一衡量模型性能的指标。还需要结合其他评估指标,如准确率,等来综合判断模型的性能。



\subsection{模型R值分析}
\begin{figure}
	\centering
	\includegraphics[width=.7\linewidth]{../../../../../pictures/rzhi.png}
	\caption{各组数据R值图}
\end{figure}

图8为模型中的回归图形,在分类问题中,回归r值本身并不直接具有解释性。但是回归分析可以帮助在训练模型时更好地理解数据的分布和特征之间的关系,在分类问题中,回归的r值,即相关系数,通常用于衡量两个变量之间的线性关系强度。然而,它本身并不直接对分类任务的性能或准确性提供明确的解释。分类问题关注的是将数据点划分为不同的类别,而r值更多地反映了连续变量之间的线性趋势。尽管如此,回归分析在分类问题的训练过程中仍然扮演着重要的角色。通过回归分析,可以更好地理解数据的分布和特征之间的关系,从而有助于构建更有效的分类模型。

该模型的训练R值为0.906:这表明在训练数据集上,模型的预测值与实际值之间具有很高的线性相关性。这可能意味着模型在训练数据上表现良好,能够很好地拟合训练数据的特征与目标变量之间的关系。测试R值为0.88:测试集是用来评估模型在未见过的数据上的泛化能力的。测试R值略低于训练R值,但仍然是一个相对较高的值,说明模型在测试数据上的预测性能也较好。验证R值为0.87:验证集用于调整模型的超参数,以避免过拟合。验证R值与测试R值相近,说明模型在验证数据上的性能也比较稳定。模型总R值为0.89:这个值可能是训练、测试和验证R值的某种加权平均,它提供了一个整体性能的概览。0.89的R值表明模型在整体上具有较好的预测能力。

\subsection{验证集结果分析}
验证数据共有42条记录,详见表16中,这些验证数据可以客观地评估模型在真实环境中的表现,可以作为验证和校准模型的重要依据。
\begin{table}[h]
	\caption{信用评价模型验证数据}
	\label{tab:papercomponents}
	\centering
	\scalebox{0.7}{
	\begin{tabular}{ccccccccccc}
		\toprule
		\multirow{2}{*}{\bfseries 序号}& \multicolumn{1}{c} {\bfseries 是否有} &multirow{2}{*}{\bfseries 是否上市} & \multicolumn{1}{c} {\bfseries 经营}& \multirow{2}{*}{\bfseries 现金比率} & \multicolumn{1}{c} {\bfseries 应收账款} & \multicolumn{1}{c} {\bfseries 销售}& \multicolumn{1}{c} {\bfseries 是否逾期}& \multicolumn{1}{c} {\bfseries 信用等级}& \multicolumn{1}{c} {\bfseries 是否逾期}& \multicolumn{1}{c} {\bfseries 信用等级} \\
		& \multicolumn{1}{c} {\bfseries 逾期记录} & & \multicolumn{1}{c} {\bfseries 负荷情况}& & \multicolumn{1}{c} {\bfseries 周转率} & \multicolumn{1}{c} {\bfseries 净利率}& \multicolumn{1}{c} {\bfseries -真实值}& \multicolumn{1}{c} {\bfseries -真实值}& \multicolumn{1}{c} {\bfseries -预测值}& \multicolumn{1}{c} {\bfseries -预测值} \\
		\midrule
	1	&	是	&	否	&	否	&	1.21	&	6.64	&	0.0662	&	否	&	A	&	否	&	A	\\
	2	&	是	&	否	&	是	&	0.22	&	0.95	&	0.2426	&	否	&	C	&	否	&	C	\\
	3	&	否	&	否	&	是	&	0.9	&	2.46	&	0.084	&	否	&	B	&	是	&	B	\\
	4	&	否	&	否	&	是	&	0.24	&	7.08	&	0.1078	&	否	&	AA	&	否	&	AA	\\
	5	&	否	&	是	&	是	&	1.37	&	7.82	&	0.2319	&	否	&	AA	&	否	&	AA	\\
	6	&	是	&	是	&	是	&	0.09	&	5.13	&	0.0098	&	是	&	B	&	是	&	B	\\
	7	&	是	&	否	&	是	&	0.89	&	10.32	&	0.0504	&	否	&	A	&	否	&	A	\\
	8	&	是	&	否	&	是	&	1.4	&	32.42	&	0.2361	&	否	&	AA	&	否	&	AAA	\\
	9	&	否	&	否	&	是	&	0.19	&	6.06	&	0.0339	&	否	&	B	&	是	&	B	\\
	10	&	是	&	是	&	是	&	0.39	&	12.31	&	0.0499	&	是	&	A	&	否	&	A	\\
	11	&	否	&	否	&	是	&	0.24	&	26.77	&	0.1089	&	否	&	AA	&	否	&	AAA	\\
	12	&	否	&	否	&	是	&	4.34	&	8.74	&	0.0144	&	否	&	A	&	否	&	AAA	\\
	13	&	否	&	否	&	是	&	0.21	&	0.75	&	0.0009	&	否	&	C	&	是	&	C	\\
	14	&	否	&	否	&	是	&	0.4	&	8.76	&	0.0334	&	否	&	A	&	否	&	A	\\
	15	&	否	&	否	&	是	&	0.54	&	62.34	&	0.0914	&	是	&	AAA	&	否	&	AAA	\\
	16	&	否	&	否	&	是	&	0.26	&	1.52	&	0.2105	&	否	&	C	&	否	&	C	\\
	17	&	否	&	是	&	是	&	0.28	&	17.26	&	0.2588	&	否	&	AA	&	否	&	AA	\\
	18	&	否	&	否	&	否	&	1.06	&	9.26	&	0.1799	&	否	&	AA	&	否	&	AA	\\
	19	&	否	&	否	&	是	&	0.16	&	1.81	&	0.2886	&	否	&	A	&	是	&	B	\\
	20	&	否	&	否	&	否	&	1.19	&	8.05	&	0.0905	&	否	&	A	&	否	&	AA	\\
	21	&	否	&	否	&	是	&	0.2	&	10.53	&	0.0922	&	否	&	A	&	否	&	A	\\
	22	&	是	&	否	&	是	&	1.08	&	17.53	&	0.1197	&	否	&	AA	&	否	&	AA	\\
	23	&	否	&	是	&	是	&	0.75	&	12.72	&	0.151	&	否	&	AAA	&	否	&	AAA	\\
	24	&	是	&	否	&	是	&	0.02	&	1.93	&	0.0059	&	是	&	C	&	是	&	B	\\
	25	&	否	&	否	&	是	&	0.22	&	5.36	&	0.0717	&	否	&	B	&	是	&	B	\\
	26	&	否	&	是	&	是	&	0.31	&	4.22	&	0.028	&	是	&	AA	&	否	&	AA	\\
	27	&	否	&	否	&	是	&	0.19	&	3.63	&	0.0303	&	否	&	A	&	否	&	B	\\
	28	&	否	&	是	&	否	&	0.07	&	11.33	&	0.1709	&	否	&	AA	&	否	&	AA	\\
	29	&	否	&	否	&	否	&	0.3	&	2.43	&	0.1045	&	是	&	B	&	否	&	B	\\
	30	&	否	&	是	&	是	&	0.95	&	11.28	&	0.2124	&	否	&	AA	&	否	&	AA	\\
	31	&	否	&	是	&	是	&	0.58	&	41.94	&	0.1814	&	否	&	AAA	&	否	&	AAA	\\
	32	&	是	&	否	&	是	&	0.19	&	13.34	&	0.2941	&	否	&	AA	&	否	&	AA	\\
	33	&	否	&	是	&	是	&	0.47	&	5.35	&	0.6895	&	否	&	AAA	&	否	&	AAA	\\
	34	&	是	&	否	&	是	&	0.63	&	5.3	&	0.0063	&	是	&	C	&	是	&	C	\\
	35	&	否	&	是	&	是	&	0.56	&	61.94	&	0.1214	&	否	&	AAA	&	否	&	AA	\\
	36	&	否	&	否	&	是	&	1.89	&	5.75	&	0.3916	&	否	&	AA	&	否	&	A	\\
	37	&	否	&	否	&	是	&	0.31	&	3.76	&	0.1686	&	否	&	A	&	否	&	A	\\
	38	&	否	&	是	&	是	&	1.25	&	21.63	&	0.0926	&	否	&	AAA	&	否	&	AAA	\\
	39	&	是	&	否	&	是	&	0.51	&	11.83	&	0.0743	&	否	&	AA	&	否	&	AA	\\
	40	&	是	&	否	&	否	&	0.69	&	13.31	&	0.2252	&	否	&	A	&	否	&	A	\\
	41	&	否	&	是	&	是	&	0.52	&	5.58	&	0.5486	&	否	&	AAA	&	否	&	AAA	\\
	42	&	否	&	否	&	是	&	0.4	&	9.95	&	0.027	&	否	&	AA	&	否	&	AA	\\	
		\bottomrule
	\end{tabular}
}
\end{table}

\begin{figure}
	\centering
	\includegraphics[width=.7\linewidth]{../../../../../pictures/yanzhengjieguo.png}
	\caption{验证结果结果对比图}
\end{figure}
可以结合图9,可以观察到,模型在预测企业未来是否会逾期方面的准确率,相较于信用等级的预测,展现出了更高的精准度。这一优势得益于输入参数与逾期风险的强关联度,这些参数在经过Logistic的精细筛选后,显现出了高度的相关性。反观信用评价的预测,虽然它是基于A公司原有指标体系计算得出的等级,但其与实际情况的关联度相对较低。尽管如此,模型所预测出的信用等级结果,却更能真实地反映企业的还款能力。图9中明显呈现出一种趋势:当模型预测某企业存在逾期风险时,其信用等级预测往往低于A公司原有的评级,这进一步印证了模型在揭示企业未来还款情况方面的有效性。
从图9中可以看出,模型在信用等级预测上准确判断了33个样本,准确率达到78.5\%。而在未来是否逾期的预测上,模型更是精准地判断了34个样本,准确率高达80.95\%。尽管数据量相对有限,但这一结果足以证明模型在信用等级预测方面具备一定的预测能力。



\iffalse
\section{A公司信用评价体系概况}

A公司的信用评价体系是在2019年由大公国际信用评级有限公司所建设,它采用了一套经过深入研究和实践验证的传统评分卡方法。这种方法,源于严谨的统计学原理与丰富的行业经验规则,为信用评估提供了可靠的基础。评分卡模型作为这一方法的核心,能够综合考量客户的多样化信息和关键指标,通过精确的数据分析和科学的权重分配,为每个客户生成一个客观公正的信用评分。这一评分不仅直观地反映了客户的信用状况,而且为A公司提供了判断客户信用等级的重要依据。通过这套体系,A公司能够更加准确地评估客户的信用风险,为公司的业务决策和风险管理提供了有力支持。

该套信用评价体系涵盖了众多评价指标,旨在全面评估客户的信用状况。然而,评分方法的一个基本假设是这些指标之间是相互独立的,即它们各自对信用评分的影响是互不干扰的。这一假设虽然简化了评分模型的构建过程,但也忽略了指标之间可能存在的相互作用和复杂关系。在实际情况中,这些指标往往相互关联,共同影响着客户的信用状况。因此,忽略这种相互作用可能导致评分模型过于简化,无法准确捕捉客户信用的真实情况。

此外,该体系在评分过程中还涉及主观的评分及标准分值制定环节。这些环节往往依赖于评估人员的经验和判断,因此存在一定的主观性和不确定性。不同的评估人员可能会根据自己的理解和偏好来制定评分标准和分值,从而导致评价结果的不一致性。这种主观性和不确定性可能会对信用评价的准确性和公正性产生负面影响,使得评价结果难以被客观验证和比较。

因此,虽然该套信用评价体系在一定程度上能够评估客户的信用状况,但由于其简化模型结构和主观评分环节的存在,评价结果可能不够准确和客观。为了提高信用评价的准确性和可靠性,需要进一步完善评分方法,考虑指标之间的相互作用和复杂关系,并减少主观性和不确定性的影响。

随着A公司的业务发展,面临更多的客户和更复杂的信用风险,该套体系计算出的客户信用等级对业务判断的准确性逐渐失去参考意义。在制定客户的应收账款账期时,新增了授信建议的环节,即由信用管理部门的人员给客户制定一个账期,最终使用的为建议过的账期,而且对比历史数据,该套体系最终得出的账期与建议的账期逐渐发生偏离,
因此需要更加准确和精细的客户信用评估工具来帮助公司做出更好的决策。

随着A公司业务的不断扩张,公司所面临的客户群体日益庞大,信用风险也呈现出更加复杂多变的态势。在这样的背景下,原有的信用评价体系逐渐显露出其局限性,计算出的客户信用等级对业务判断的参考价值逐渐减弱。为了更好地适应业务发展的需求,A公司在制定客户的应收账款账期时,新增了授信建议环节。这一环节由信用管理部门的专业人员根据客户的具体情况,结合市场环境和公司策略,为客户制定一个合理的账期。然而,在实际操作过程中,发现该套体系最终得出的账期与信用管理部门建议的账期逐渐产生了偏离。

这种偏离不仅影响了公司对应收账款管理的准确性,也增加了业务决策的风险。为了解决这个问题,A公司急需寻找更加准确、精细的客户信用评估工具。这种工具需要能够综合考虑客户的各项信息和指标,深入挖掘指标之间的相互作用和复杂关系,从而更准确地评估客户的信用状况。

通过引入更加科学的客户信用评估工具,A公司可以更加精准地把握客户的信用风险,提高业务决策的质量和效率。这将有助于公司优化应收账款管理,降低坏账风险,进而提升整体盈利能力和市场竞争力。

\fi

\subsection{实证分析结论}

 信用等级预测方面,模型在信用等级预测方面表现出了较高的准确性。通过对比测试集及验证集上的预测结果与实际信用等级,本研究发现模型能够准较确地区分不同信用等级的客户,为业务提供有力的决策支持。
 
 在预测客户未来是否会逾期方面,模型同样展现出了良好的性能。通过对客户的历史还款记录和其他相关指标的学习,模型能够有效地识别出具有逾期风险的客户,为业务度对客户的风险管理提供了重要依据。
  该模型在信用等级预测和未来是否逾期预测方面表现良好。虽然两个预测结果的准确率都较高,但未来是否逾期的预测准确率略高于信用等级预测。这可能是因为逾期与否的预测与实际的还款行为更为直接相关,模型在捕捉这种直接关联上可能更具优势。
 
 虽然该模型在当前的数据集上表现出了较好的性能,但还需要注意,模型的性能可能会受到多种因素的影响,如数据集的分布、指标的选择、模型的参数设置等。因此,在未来的研究中,可以进一步探索如何优化这些因素,以提高模型的性能。
\section{本章小结}
本章基于第三章构建的信用评价模型,利用A公司的企业客户数据进行了进行了模型训练及检验。首先将收集到的数据进行处理,利用处理后的数据对模型进行了反复的训练。通过不断调整模型的参数,努力寻找最优的模型配置,以最大程度地提高模型的预测准确性。在训练过程中,采用了交叉验证等方法,对模型进行了充分的验证和评估,确保了模型的稳定性和泛化能力。
完成训练后,使用验证数据将模型的预测结果与实际数据进行了对比和分析。通过计算模型的准确率、拟合度等关键指标,全面评估了模型的预测性能。