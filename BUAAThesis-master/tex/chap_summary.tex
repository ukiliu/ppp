% !TeX root = ../Template.tex
% 总结
\summary

信用等级是对企业信用能力和还款意愿的综合评估,等级高低可以反映企业的信用状况和履约能力。信用等级高并不意味着企业客户一定不会逾期。尽管信用等级高的企业客户在信用评估中表现出色,具有较低的信用风险和良好的还款能力,但这并不意味着他们完全没有违约或逾期的可能性。在实际业务中,各种不可预见的情况和突发事件可能导致企业客户无法按时偿还债务。因此,高信用等级并不绝对保证企业不会出现逾期的情况。

1.研究结论

对于以上问题,本研究在构建信用评价模型时,分别在信用等级和未来是否会逾期这两个方面进行了预测。
研究结果表明:基于经营负荷情况、现金比率、应收账款周转率、是否有逾期记录、是否上市以及销售净利率这六个指标构建的BP神经网络方法可以有效预测企业客户未来是否会发生逾期,使用Matlab编程语言,借助计算机工具可以方便地完成BP网络模型的算法设计以及数据运算,建立起预测企业客户信用的模型,更加准确地评估客户的信用风险,为公司的业务决策和风险管理提供了有力支持。

%在指标筛选方面,本文在了财务指标、业务指标、企业资质相关指标这三个一级指标下,选取了23个指标来筛选客户%未来是否会逾期的影响因素,采用二元Logistic回归分析方法,Logistic模型的输出为客户未来6个月是否会逾期%%,利用A公司的客户的历史数据作来做模型分析,最终筛选出6个影响指标,建立起客户信用评价指标。在经过模型验%证之后,得出该套指标体系能较好的反映出A公司企业客户未来是否会逾期。在构建评价模型方面,使用BP神经网络%构建立了适用于A公司的企业客户信用评价模型,输入为构建起的6个指标体系,输出为是客户信用等级和未来是否会%逾期。构建了三层的神经网络,经过多次训练与调参,得出一套拟合程度较好的网络模型。最后基于该模型,使用A公%司近半年的客户数据作为验证,


%本研究中同样存在不足之处,因为本研究是基于已有结果去预测未来的数据,已有结果包括未来是否逾期和信用评价%等级,这些数据均采用A公司信用评价系统的相关数据,数据不具有公开性,而且对于未来是否有逾期是基于A公司现%有的业务判断逻辑来定义;信用评价等级因为多数客户的信用等级无法获取,所以使用的是信用管理系统现有的评价%等级。
%使用特定公司的业务数据不具有广泛的代表性,无法全面反映整个行业或市场的状况,所以研究的评价模型有一定的%局限性。

2.不足及改进方向

在数据量方面,目前本研究数据来源于A公司的企业客户,收集的数据量较为有限,对于构建和验证BP神经网络模型而言,这样的数据量确实存在一定的局限性。在数据量不够充分的情况下,神经网络可能无法充分捕捉到数据的内在规律和特征,从而导致模型的泛化能力受限,预测结果不够精准。

模型输出中使用的信用等级数据来源具有局限性。信用等级采用A公司信用评价系统得出的结果。该信用等级仅反映了A公司内部的业务逻辑和评价标准,而这些标准和逻辑可能并不适用于其他公司或行业。使用特定公司的业务数据作为研究基础,其代表性和普遍性难免受到质疑。A公司的业务模式和评价标准可能与其他公司存在显著差异,因此其数据可能无法全面反映整个行业或市场的真实状况。这使得研究所得出的评价模型在推广和应用时存在一定的局限性,难以直接应用于其他公司或行业。

针对以上不足可以结合行业内的其他公司或引入其他信用评价机构的信用评价数据,以获取更广泛、更全面的信用信息。通过扩大数据集的规模,为BP神经网络提供更多的学习样本,从而提高模型的泛化能力。

 